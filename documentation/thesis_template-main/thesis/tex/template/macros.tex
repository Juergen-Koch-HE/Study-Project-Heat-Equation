\newcommand{\marktext}[1]{\textcolor{red}{#1}}
\newcommand{\redbf}[1]{\textbf{\textcolor{red}{#1}}}
\newcommand{\theadstyle}{\bfseries\color{blue!60!black}}
\renewcommand\arraystretch{1.35}

\newcommand{\todoinline}[1]{\todo[inline, color=blue!40]{#1 \textit{(\studentshortname)}}}
\newcommand{\todonote}[1]{\todo[color=blue!40]{#1 \textit{(\studentshortname)}}}
\newcommand{\review}{\todo[inline, color=green!40]{REVIEW}}
\newcommand{\feedback}[1]{\todo[inline, color=purple!40]{FEEDBACK #1}}
\newcommand{\suggest}[2]{
        \textcolor{red}{\sout{#1}}
        \textcolor{olive}{#2}
}



\newtheorem{theorem}{Theorem}[chapter]
\newtheorem{lemma}[theorem]{Lemma}


\newtheoremstyle{goalStyle}% name of the style to be used
  {\topsep}% measure of space to leave above the theorem. E.g.: 3pt
  {\topsep}% measure of space to leave below the theorem. E.g.: 3pt
  {\itshape}% name of font to use in the body of the theorem
  {}% measure of space to indent
  {\bfseries}% name of head font
  {}% punctuation between head and body
  {5pt plus 1pt minus 1pt}% space after theorem head; " " = normal interword space
  {\thmname{#1 }\thmnumber{(G#2)}\thmnote{ (#3)}}
\theoremstyle{goalStyle}
\newtheorem{goal}{Goal}
\newcommand{\refgoal}[2]{\hyperref[#1]{(#2 G\ref{#1})}}