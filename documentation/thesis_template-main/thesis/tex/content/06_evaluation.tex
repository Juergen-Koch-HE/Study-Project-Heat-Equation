\chapter{Evaluation}
Frage: Ist das, was du gebaut hast, sinnvoll / korrekt?

Hier bewertest du:

Stimmt die Simulation mit der Theorie überein?

Verhalten bei großen Zeiten?

Plausibilität (Abkühlung, Glättung)

didaktischer Nutzen für Studierende

Vergleichen, bewerten, einordnen.

Das Labor zur Wärmeleitungsgleichung bietet einen guten Einstieg und sinnvollen Einstieg im Rahmen des Moduls Modellbildung und Simulation, da es die Theorie erklärt, näher bringt und zu guter letzt graphisch ansprechend darstellt. 

Aufgrund der vereinfachten Laborbedingungen, wie lediglich die Betrachtung der Wärmeleitung in einem eindimensionalen Stabelement oder auch den unrealistischen Anfangs- sowie Randbedingungen ist das Labor nicht realitätsgetreu. Weitere nicht berücksichtigte Parameter sind keine Wärmequelle, so wird angenommen, dass das eindimensionale Stabelement nur abkühlt, aber keine Wärmequelle/Wärmezufuhr besitzt.  Diese vereinfachungen der Bedingungen helfen allerdings dem leichteren Einstieg in die Materie und rücken den didaktischen Nutzen für Studierende in den Fokus.

Trotz den vereinfachten Laborbedingungen können wichtige Beobachtungen bei den Simulationen gemacht werden, wie etwa das Abkühlungsverhalten und die Glättung von Wärmespitzen.