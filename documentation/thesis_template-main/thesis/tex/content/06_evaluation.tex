\chapter{Evaluation}
Das Labor zur Wärmeleitungsgleichung bietet einen guten und sinnvollen Einstieg im Rahmen des Moduls Modellbildung und Simulation, indem es die theoretische Herleitung aus dem Energieerhaltungssatz mit einer anschaulichen grafischen Darstellung verknüpft.

Aufgrund der vereinfachten Laborbedingungen, wie lediglich der Betrachtung der Wärmeleitung in einem eindimensionalen Stabelement oder auch den unrealistischen Anfangs- sowie Randbedingungen, ist das Labor nicht realitätsgetreu. Weitere nicht berücksichtigte Parameter sind keine Wärmequelle, so wird angenommen, dass das eindimensionale Stabelement nur abkühlt, aber keine Wärmequelle/Wärmezufuhr besitzt.  Diese Vereinfachungen der Bedingungen helfen allerdings dem leichteren Einstieg in die Materie und rücken den didaktischen Nutzen für die Studierenden in den Fokus.

Trotz der vereinfachten Laborbedingungen können wichtige Beobachtungen bei den Simulationen gemacht werden, wie etwa das Abkühlungsverhalten und die Glättung von Wärmespitzen. Zudem werden Fourier-Reihen eingeleitet, mit denen sich viele Funktionen darstellen lassen können und so einen allgemeinen Nährwert bieten. In dem Kontext der Fourier-Reihen wurden dann auch zu beachtende Phänomene aufgezeigt, wie das Gibbsche Phänomen und die Überschwingungen, sowie dessen Glättung.

Insbesondere helfen die vereinfachten Laborbedingungen dem Studierenden dabei, sich mit der mathematischen Thematik auseinanderzusetzen, da zu Beginn des Themas nicht direkt der Fokus darauf liegt, schwere und sehr komplexe Gleichungen zu lösen. Stattdessen steht im Vordergrund, dass der Studierende die physikalischen Hintergründe der Wärmeleitungsgleichung begreifen kann, um später auch komplexere Gleichungen lösen zu können.
Das Thema bietet eine solide Grundlage, auf der weitere Projekte aufbauen können, zum Beispiel eine 2D-Simulation der Wärmeleitungsgleichung.
Durch die interaktive Komponente, das Berechnen jedes Schrittes im Labor und das Anschauen des Modell-Videos, sollte ein einfacher Einstieg gelingen.