\chapter{Einleitung}


\section{Ziel des Studienprojekts}\label{sec:goals}
Ziel des Studienprojekts “Simulation der Wärmeleitungsgleichung in Festkörpern”, ist es, den Teilnehmern des Kurses “Modellbildung und Simulation” den Vorgang der Wärmeleitung in einem Festkörper anhand eines interaktiven Aufgabensets in einem Labor greifbar verständlich zu machen. Die Teilnehmer des Kurses sollen in diesem Labor erst die theoretischen Grundlagen der Wärmeleitungsgleichung für ein eindimensionales Stabelement erhalten. Anschließend können Sie sich dann selbstständig Stück für Stück durch die einzelnen Aufgabenteile durcharbeiten, welche das numerische lösen beinhalten, programmieren in Python, was dann zu einer graphischen Darstellung führt. Am Ende des Labors soll jeder Teilnehmer die Möglichkeit gehabt haben, einmal alle Teile einer homogenen Wärmeleitungsgleichung in 1D berechnet zu haben und das Ergebnis visuell in einer Heatmap simulieren zu können. So soll der Lerneffekt durch die “Learning-by-doing” Methode eintreten und das Interesse der Teilnehmer für Modellbildung und Simulation gestärkt werden.

Ziel des Projekts ist die Modellierung und Simulation der Wärmeleitungsgleichung in Festkörpern in 1D. Die Simulation wird in der Programmiersprache Python implementiert und dient der Visualisierung der zeitlichen und räumlichen Temperaturverteilung innerhalb eines festen Körpers.

Die entwickelte Anwendung soll im Rahmen des Labors Modellbildung und Simulation eingesetzt werden. Hierfür wird eine geeignete Aufgabenstellung konzipiert, welche die mathematische Modellbildung sowie die Interpretation der Simulationsergebnisse praxisnah vermittelt.


\section{Aufbau Jupyter Notebook}
Nachfolgend der Aufbau des Jupyter Notebook welches Studierenden in dem Fach “Modellbildung und Simulation” neben anderen Laboren bereit steht.

\begin{table}[h]
\centering
\renewcommand{\arraystretch}{1.3}
\begin{tabularx}{\textwidth}{|c|p{4cm}|X|}
\hline
\textbf{Nr.} & \textbf{Thema} & \textbf{Beschreibung}\\
\hline
0 &
Imports &
Festlegung der Ziele, Sammlung fachlicher Grundlagen sowie Sichtung vergleichbarer Projekte und Werkzeuge.  \\
\hline
1 &
Modelling &
Als Einleitung zum Thema wird ein Video eingebunden, das das Verhalten des Materials beschreibt, wenn es an einer Seite durch eine Wärmequelle erhitzt wird.  \\
\hline
2 &
Derivation of the heat equation &
Herleitung aus der Energiebilanz in einem infinitesimalen Volumenelement herleiten  \\
\hline
3 &
Numerical treatment &
Erstellung einer Übungsaufgabe für Studierende, bei der der Temperaturleitkoeffizient berechnet wird.  \\
\hline
3.1 &
Thermal diffusity &
Erstellung einer Übungsaufgabe für Studierende, bei der der Temperaturleitkoeffizient berechnet wird.  \\
\hline
3.2 &
Partial Differential Equations &
Erstellung einer Übungsaufgabe für Studierende, bei der der Temperaturleitkoeffizient berechnet wird.  \\
\hline
3.2.1 &
Boundary Conditions &
Erstellung einer Übungsaufgabe für Studierende, bei der der Temperaturleitkoeffizient berechnet wird.  \\
\hline
3.3 &
Fourier coefficient &
Erstellung einer Übungsaufgabe für Studierende, bei der der Temperaturleitkoeffizient berechnet wird.  \\
\hline
4 &
Implementation \& Visualization &
Lösung der Wärmeleitungsgleichung für einen eindimensionalen Stab ohne Wärmequelle. Explizite Lösung mit Anfangs- und Randbedingungen.  \\
\hline
4.1 &
Simple sine mode - Diagram &
Lösung der Wärmeleitungsgleichung für einen eindimensionalen Stab ohne Wärmequelle. Explizite Lösung mit Anfangs- und Randbedingungen.  \\
\hline
4.2 &
Fourier sine mode - Diagram &
Lösung der Wärmeleitungsgleichung für einen eindimensionalen Stab ohne Wärmequelle. Explizite Lösung mit Anfangs- und Randbedingungen.  \\
\hline
\end{tabularx}
\caption{Projektphasen und Meilensteine}
\label{tab:projektphasen}
\end{table}



