\chapter{Einleitung}


\section{Ziel des Studienprojekts}\label{sec:goals}
Ziel des Studienprojekts “Simulation der Wärmeleitungsgleichung in Festkörpern”, ist es, den Teilnehmern des Kurses “Modellbildung und Simulation” den Vorgang der Wärmeleitung in einem Festkörper anhand eines interaktiven Aufgabensets in einem Labor greifbar verständlich zu machen. Die Teilnehmer des Kurses sollen in diesem Labor erst die theoretischen Grundlagen der Wärmeleitungsgleichung für ein eindimensionales Stabelement erhalten. Anschließend können sie sich dann selbstständig Stück für Stück durch die einzelnen Gleichungsteile durcharbeiten, welche das analytische lösen beinhalten, programmieren in Python, was dann zu einer graphischen Darstellung führt. Am Ende des Labors soll jeder Teilnehmer die Möglichkeit gehabt haben, einmal alle Teile einer homogenen Wärmeleitungsgleichung in 1D berechnet zu haben und das Ergebnis visuell in einer Heatmap simulieren zu können. So soll der Lerneffekt durch die “Learning-by-doing” Methode eintreten und das Interesse der Teilnehmer für Modellbildung und Simulation gestärkt werden.

Ziel des Projekts ist die Modellierung und Simulation der Wärmeleitungsgleichung in Festkörpern in 1D. Die Simulation wird in der Programmiersprache Python implementiert und dient der Visualisierung der zeitlichen und räumlichen Temperaturverteilung innerhalb eines festen Körpers.

Die entwickelte Anwendung soll im Rahmen des Labors Modellbildung und Simulation eingesetzt werden. Hierfür wird eine geeignete Aufgabenstellung konzipiert, welche die mathematische Modellbildung sowie die Interpretation der Simulationsergebnisse praxisnah vermittelt.


\section{Aufbau des Labors}
Nachfolgend wird der Aufbau des Labors "Simulation der Wärmeleitungsgleichung in Festkörpern" in einem Jupyter Notebook beschrieben, das den Studierenden im Fach “Modellbildung und Simulation” neben anderen Laboren bereitsteht.

\begin{table}[h]
\centering
\renewcommand{\arraystretch}{1.3}
\begin{tabularx}{\textwidth}{|c|p{4cm}|X|}
\hline
\textbf{Nr.} & \textbf{Thema} & \textbf{Beschreibung}\\
\hline
0 &
Imports &
Einrichten der Arbeitsumgebung durch das Importieren von später benötigten Bibiliotheken.  \\
\hline
1 &
Modelling &
Als Einleitung zum Thema wird ein Video eingebunden, das das Verhalten des Materials beschreibt, wenn es an einer Seite durch eine Wärmequelle erhitzt wird. Dies soll als Motivation und Modell dienen.  \\
\hline
2 &
Derivation of the heat equation &
Herleitung der Wärmeleitungsgleichung aus der Energiebilanz und Fouriers Gesetz für ein infinitesimales Volumenelement.  \\
\hline
3 &
Analytical treatment &
\\
\hline
3.1 &
Thermal diffusity &
Die erste Aufgabe für Studierende, bei dem der Temperaturleitkoeffizient berechnet wird.  \\
\hline
3.2 &
Partial Differential Equations &
Einordnung als partielle Differenzialgleichung, einführung der Modellannahme, in dem sich das eindimensionale Stabelement befindet. Bekanntmachung der eindimensionalen Wärmeleitungsgleichung ohne Wärmequelle sowie Rand- und Anfangsbedingungen. \\
\hline
3.2.1 &
Boundary Conditions &
Anwenden der Randbedingungen bei der vorausgehenden Lösung des Ortsteils. \\
\hline
3.3 &
Fourier coefficient &
Berechnung des Koeffizienten $b_k$ über gegebene Integral Formel, damit die Summe der Sinusmoden die Anfangsbedingung möglichst gut nachbildet.  \\
\hline
4 &
Implementation \& Visualization &
\\
\hline
4.1 &
Simple sine mode &
Umsetzung der analytischen Lösung mit nur einer Sinuswelle um eine erste einfache Simulation anhand eines Funktionsgraphen aufzuzeigen. Dadurch können die Randbedingungen sowie das exponentielle abklingen beobachtet werden.  \\
\hline
4.2 &
Fourier sine mode &
Lösung der Wärmeleitungsgleichung für einen eindimensionalen Stab ohne Wärmequelle. Explizite Lösung mit Anfangs- und Randbedingungen.  \\
\hline
\end{tabularx}
\caption{Projektphasen und Meilensteine}
\label{tab:projektphasen}
\end{table}



