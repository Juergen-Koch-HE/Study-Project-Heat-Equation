\chapter{Einleitung}
This \LaTeX template is just a guideline to get started with your thesis, it is not a one fits all solution.
For further ideas and tips, please refer to additional literature, like ``How to Write a Better Thesis'' from Evans et al.~\cite{DBLP:books/sp/EvansGZ14}.

\section{Motivation}
Die Verfasser dieser Dokumentation wurden während ihres Studienwegs von Professor Koch begleitet, der sie dadurch dazu motivierte, sich mit dem Thema Wärmeleitgleichung in Festkörpern auseinanderzusetzen. Durch seinen Kurs Modellbildung und Simulation hatten sie ihre ersten praktischen Berührungspunkte mit der Programmiersprache Python und der Arbeitsumgebung Jupiter-Notebook. Mit diesen Werkzeugen vermittelte Professor Koch die wissenschaftlichen Phänomene nicht nur theoretisch, sondern vor allem anschaulich und nachvollziehbar über Simulationen und Videos. Dies bildete die wissenschaftliche Grundlage für die Verfasser. Diese Werkzeuge bilden auch die Basis dieses Projekts. Dadurch das sie eine solch nachhaltige Lernerfahrung machen durften, war es für die Verfasser klar, dass sie mit diesem Projekt dem Erfolg des Kurses einen kleinen Tribut zollen möchten und den zukünftigen Teilnehmer eine interaktive Lernerfahrung ermöglichen wollen, wie auch sie selbst sehr genossen haben.

\section{Ziel des Studienprojekts}\label{sec:goals}
Ziel des Studienprojekts Simulation der Wärmeleitgleichung in Festkörpern, war es den Teilnehmern des Kurses Modelbildung und Simulation den Vorgang der Wärmevertei-lung in einem Festkörper anhand eines interaktiven Aufgabensets in einem Labor greif-bar verständlich zu machen. Die Teilnehmer des Kurses sollen in diesem Labor erst die Theoretischen Grundlagen erhalten und können sich dann selbstständig Stück für Stückdurch die Aufgabenteile erarbeiten. Am Ende des Labors soll jeder Teilnehmer die Möglichkeit gehabt haben einmal alle Teile einer homogenen Wärmeleitgleichung be-rechnet zu haben und das Ergebnis visuelle in einer Heatmap simulieren könne. So soll der Lerneffekt durch die “Learning by doing” Methode eintreten und das Interesse der Teilnehmer für die Theoretischephysik gestärkt werden.

Ziel des Projekts ist die Modellierung und Simulation der Wärmeleitungsgleichung in Festkörpern. Die Simulation wird in der Programmiersprache Python implementiert und dient der Visualisierung der zeitlichen und räumlichen Temperaturverteilung innerhalb eines festen Körpers.

Die entwickelte Anwendung soll im Rahmen des Labors Modellbildung und Simulation eingesetzt werden. Hierfür wird eine geeignete Aufgabenstellung konzipiert, welche die mathematische Modellbildung, sowie die Interpretation der Simulationsergebnisse praxisnah vermittelt.

Zusätzlich ist vorgesehen, die fertige Anwendung auf einem vorhandenen Hardware-Cluster der Hochschule bereitzustellen, um eine performante und skalierbare Ausführung zu ermöglichen. Dabei werden gegebenenfalls bestehende Infrastruktur- und Zugriffskonzepte berücksichtigt.





\section{Structure}
Finally, you should quickly highlight the structure of your thesis and the solution, but keep it short and leave out details.
It is important that you already present a little insight into your structure, as the reader can follow easier through background and related work.