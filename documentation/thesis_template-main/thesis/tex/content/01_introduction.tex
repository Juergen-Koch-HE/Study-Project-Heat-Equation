\chapter{Einleitung}


\section{Ziel des Studienprojekts}\label{sec:goals}
Ziel des Studienprojekts “Simulation der Wärmeleitungsgleichung in Festkörpern”, ist es, den Teilnehmern des Kurses “Modellbildung und Simulation” den Vorgang der Wärmeverteilung in einem Festkörper anhand eines interaktiven Aufgabensets in einem Labor greifbar verständlich zu machen. Die Teilnehmer des Kurses sollen in diesem Labor erst die theoretischen Grundlagen erhalten und können sich dann selbstständig Stück für Stück durch die Aufgabenteile erarbeiten. Am Ende des Labors soll jeder Teilnehmer die Möglichkeit gehabt haben, einmal alle Teile einer homogenen Wärmeleitungsgleichung berechnet zu haben und das Ergebnis visuell in einer Heatmap simulieren zu können. So soll der Lerneffekt durch die “Learning-by-doing” Methode eintreten und das Interesse der Teilnehmer für Modellbildung und Simulation gestärkt werden.

Ziel des Projekts ist die Modellierung und Simulation der Wärmeleitungsgleichung in Festkörpern. Die Simulation wird in der Programmiersprache Python implementiert und dient der Visualisierung der zeitlichen und räumlichen Temperaturverteilung innerhalb eines festen Körpers.

Die entwickelte Anwendung soll im Rahmen des Labors Modellbildung und Simulation eingesetzt werden. Hierfür wird eine geeignete Aufgabenstellung konzipiert, welche die mathematische Modellbildung sowie die Interpretation der Simulationsergebnisse praxisnah vermittelt.


\section{Aufbau Jupyter Notebook}
Nachfolgend der Aufbau des Jupyter Notebook welches Studierenden in dem Fach “Modellbildung und Simulation” neben anderen Laboren bereit steht.

\begin{table}[h]
\centering
\renewcommand{\arraystretch}{1.3}
\begin{tabularx}{\textwidth}{|c|p{4cm}|X|c|}
\hline
\textbf{Nr.} & \textbf{Meilenstein} & \textbf{Beschreibung} & \textbf{Geplanter Zeitraum} \\
\hline
M1 &
Projektvorbereitung und Literaturrecherche &
Festlegung der Ziele, Sammlung fachlicher Grundlagen sowie Sichtung vergleichbarer Projekte und Werkzeuge. &
Anfang Oktober \\
\hline
M2 &
Labor: Modell &
Als Einleitung zum Thema wird ein Video eingebunden, das das Verhalten des Materials beschreibt, wenn es an einer Seite durch eine Wärmequelle erhitzt wird. &
Anfang Oktober \\
\hline
M3 &
Labor: Herleitung der Wärmeleitungsgleichung &
Herleitung aus der Energiebilanz in einem infinitesimalen Volumenelement herleiten &
Oktober \\
\hline
M4 &
Labor: Aufgabe zum Temperaturleitkoeffizient &
Erstellung einer Übungsaufgabe für Studierende, bei der der Temperaturleitkoeffizient berechnet wird. &
Oktober \\
\hline
M5 &
Labor: Aufgabe zur Wärmeleitungsgleichung &
Lösung der Wärmeleitungsgleichung für einen eindimensionalen Stab ohne Wärmequelle. Explizite Lösung mit Anfangs- und Randbedingungen. &
November \\
\hline
M7 &
Labor: Implementierung \& Visualisierung &
Implementierung von Python Funktionen sowie einer Visualisierung durch Simulationen &
November - Januar \\
\hline
M9 &
Dokumentation und Präsentation &
Erstellung des wissenschaftlichen Abschlussberichts, Darstellung der Ergebnisse sowie Vorbereitung der Abschlusspräsentation. &
Februar \\
\hline
\end{tabularx}
\caption{Projektphasen und Meilensteine}
\label{tab:projektphasen}
\end{table}



