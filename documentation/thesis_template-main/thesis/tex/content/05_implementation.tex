\chapter{Implementation}

\section{Technologiestack und Entwicklungsumgebung}
Aufgrund gegebenen Laboren, wurde auf für dieses Labor auf Jupyter Notebook gesetzt als Webanwendung in der Python programmiert werden kann. Weitere Gründe für Jupyter Notebook sind die Einfachheit, in welcher jeder dieses Verwenden kann und keine lokale konfiguration stattfinden muss, die didaktisch gute Kombination von Python-Code, Text und LaTeX-Formeln, sowie das darstellen von Plots.

Als Programmiersprache wurde Python gewählt, da diese sich hervorragend für Berechnungen eignet, besonders in Kombination mit den verwendeten Bibliotheken NumPy und Matplotlib.  

\section{Aufbau des Labors}
Der Aufbau des Labors lehnt sich an dem Aufbau von dem Labor des Roboter-Arms. Das bedeutet zunächst folgt eine Einleitung in das Thema mit einem kleinen Video, gefolgt von einer mathematischen Einleitung in das Thema, ein paar Aufgaben, welche dann zum weiteren Verständnis mithilfe von Python und den genannten Bibliotheken visualisiert als Simulationen. Aufgrund dieser Simulationen soll tieferes Verständnis für die Wärmeleitungsgleichung erzeugt werden. 


Frage: Wie hast du das Ganze praktisch umgesetzt?

Jetzt wird’s technisch:

Bei dir z. B.:

Python

Jupyter Notebook

NumPy / Matplotlib

Struktur des Labors

Aufbau der Simulation

Visualisierung (Heatmap)

🧠 Hier darf Code rein – aber erklärend, nicht dump.
