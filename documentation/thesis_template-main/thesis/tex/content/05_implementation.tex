\chapter{Implementation}

\section{Technologiestack und Entwicklungsumgebung}
Aufgrund gegebenen Laboren, wurde auf für dieses Labor auf Jupyter Notebook gesetzt als Webanwendung in der Python programmiert werden kann. Weitere Gründe für Jupyter Notebook sind die Einfachheit, in welcher jeder dieses Verwenden kann und keine lokale konfiguration stattfinden muss, die didaktisch gute Kombination von Python-Code, Text und LaTeX-Formeln, sowie das darstellen von Plots.

Als Programmiersprache wurde Python gewählt, da diese sich hervorragend für Berechnungen eignet, besonders in Kombination mit den verwendeten Bibliotheken NumPy und Matplotlib.  

\section{Aufbau des Labors}
Der Aufbau des Labors lehnt sich an dem Aufbau von dem Labor des Roboter-Arms. Das bedeutet zunächst folgt eine Einleitung in das Thema mit einem kleinen Video, gefolgt von einer mathematischen Einleitung in das Thema, ein paar Aufgaben, welche dann zum weiteren Verständnis mithilfe von Python und den genannten Bibliotheken visualisiert als Simulationen. Aufgrund dieser Simulationen soll tieferes Verständnis für die Wärmeleitungsgleichung erzeugt werden. 

Das Labor für das Modul Modellbildung und Simulation soll den Studenten schrittweise die Lösung der Wärmeleitgleichung nach dem Produktverfahren nähergebracht werden.
\begin{enumerate}
\item Imports: Sollen gleich bleiben, wie bei den anderen Laboren.

\item Modelling: Als Einleitung zum Thema, wird ein Video eingefügt dieses Video beschreibt das Verhalten eines eindimensionalen Stabelements, wenn es an einer Seite durch eine Wärmequelle erhitzt wird. 

\item Derivation of the heat equation: Darauf folgt die Herleitung der Wärmeleitungsgleichung aus dem Energieerhaltungs- Fouriers- und Gaußgesetz. Danach muss der Student seine erste Aufgabe lösen diese beinhaltet das Berechnen des Temperaturleitkoeffizienten. Die notwendigen Informationen sind durch Wikipedia Links bereitgestellt.

\item Implementation:
\item Visualization:

\item Lösen der Wärmeleitgleichung für Stab in 1D ohne Wärmequelle. Explizite Lösung mit Anfangs- und Randbedingungen.

\item Simulation/Diskretisierung
\end{enumerate}




\subsection{Werkzeuge und Infrastruktur}
Die technische Umsetzung erfolgt in der Programmiersprache Python unter Verwen-dung von:
-	Jupyter Notebook für interaktive Entwicklung
-	NumPy für numerische Berechnungen
-	Matplotlib für Visualisierung der Ergebnisse
-	Pandas für die strukturierung der berechneten ergebnise
-	Math für das nutzen von variablen wie pi, cos, sin, ...
Die Anwendung soll abschließend auf einem Hochschul-Cluster bereitgestellt werden. Die Kompatibilität mit der dortigen Infrastruktur (Python-Version, Bibliotheken, evtl. Containerisierung) wird im Projektverlauf berücksichtigt.

\subsection{Aufbau Labor}

\begin{figure}[htbp]
  \centering
  \includegraphics[width=0.8\textwidth]{content/figures/SingleSineModeDiagram.png}
  \caption{Zeitliche und räumliche Temperaturverteilung im eindimensionalen Stab}
  \label{fig:heat-simulation}
\end{figure}

\begin{figure}[htbp]
  \centering
  \includegraphics[width=0.8\textwidth]{content/figures/SingleSineModeHeatmap.png}
  \caption{Zeitliche und räumliche Temperaturverteilung im eindimensionalen Stab}
  \label{fig:heat-simulation1}
\end{figure}

\begin{figure}[htbp]
  \centering
  \includegraphics[width=0.8\textwidth]{content/figures/TruncatedFourierSineSolution.png}
  \caption{Zeitliche und räumliche Temperaturverteilung im eindimensionalen Stab}
  \label{fig:heat-simulation2}
\end{figure}
\begin{figure}[htbp]
  \centering
  \includegraphics[width=0.8\textwidth]{content/figures/InitialConditionApproximation.png}
  \caption{Zeitliche und räumliche Temperaturverteilung im eindimensionalen Stab}
  \label{fig:heat-simulation3}
\end{figure}

\begin{figure}[htbp]
  \centering
  \includegraphics[width=0.8\textwidth]{content/figures/FourierSineModeHeatmapBeginning.png}
  \caption{Zeitliche und räumliche Temperaturverteilung im eindimensionalen Stab}
  \label{fig:heat-simulation4}
\end{figure}

\begin{figure}[htbp]
  \centering
  \includegraphics[width=0.8\textwidth]{content/figures/FourierSineModeHeatmapLater.png}
  \caption{Zeitliche und räumliche Temperaturverteilung im eindimensionalen Stab}
  \label{fig:heat-simulation5}
\end{figure}

\FloatBarrier

\subsection{Gibbsches Phänomen in der Simulation}
aaaaaaaaaaaaaaaaaaaaaaaaaaaaaaaaaaa
\subsection{Überschwringungen Glätten}
aaaaaaaaaaaaaaaaaaaaaaaaaaaaaaaaaaa
