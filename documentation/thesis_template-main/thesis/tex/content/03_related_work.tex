\chapter{Herleitung der Wärmeleitungsgleichung}
Die Wärmeleitungsgleichung lässt sich auch direkt aus der Energiebilanz in einem infinitesimalen Volumenelement herleiten \cite[S.~15--28]{langemann2018so}.
Dabei wird der Energieerhaltungssatz mit dem Fourierschen Gesetz kombiniert, um die partielle Differentialgleichung für die Temperaturentwicklung abzuleiten.

\subsubsection{Energieerhaltung}
Betrachten wir ein kleines Stück eines Stabes im Intervall \([x, x + \Delta x]\).
Die in diesem Stück gespeicherte thermische Energie \(E\) ist proportional zur Temperatur \(u(x,t)\):
\[
E = \rho \, c_p \, u(x,t) \, \Delta x
\]


\subsubsection{Wärmestrom nach Fourier}
Der Wärmestrom \(I\) wird durch das Fouriersche Gesetz beschrieben:
\[
I = -k \, u_x
\]

\subsubsection{Netto-Wärmefluss und Energiebilanz}
Der Netto-Wärmefluss in das Intervall \([x, x + \Delta x]\) ergibt sich aus der Differenz der Wärmeströme an den Grenzen:
\[
I(x) - I(x + \Delta x) \approx -I_x \Delta x
\]
Nach dem Energieerhaltungssatz ist die zeitliche Änderung der Energie im Volumenelement gleich dem Netto-Wärmefluss:
\[
\rho c_p u_t \Delta x = - I_x \Delta x
\]
Durch Kürzen von \(\Delta x\) erhält man:
\[
\rho c_p u_t = - I_x
\]

\subsubsection{Einsetzen des Fourier'schen Gesetzes}
Setzt man das Fouriersche Gesetz für \(I\) ein, ergibt sich:
\[
I = -k u_x
\]

\[
I_x = - (k u_x)_x
\]

Damit wird die Energiebilanz zu:
\[
\rho c_p u_t = (k u_x)_x
\]

\subsubsection{Vereinfachung für homogene Materialien}
Für homogene Materialien (konstante Wärmeleitfähigkeit \(k\)) vereinfacht sich die Gleichung zu:
\[
u_t = \frac{k}{\rho c_p} u_{xx}
\]
Die Konstante
\[
a^2 = \frac{k}{\rho c_p}
\]
wird als thermische Diffusivität bezeichnet. Die Wärmeleitungsgleichung lautet somit:
\[
u_t(x,t) = a^2 \cdot u_{xx}(x,t)
\]



!!!!!!!!!!!!!Alles kann man lschen eigentlich 
Als äquivalent geltende Notationen sind:

$$
u_t =a^2 \cdot u_{xx}
$$

$$
u_t = a^2 u_{xx}
$$


bish hier  !!!!!!!!!!!!!!!!!!!!!!!!!!!!!!!!!!!!

%So einfach ist Mathematik, Partielle Differentialgleichungen für Anwender. Springer, Abschnitte 2.1--2.3.

