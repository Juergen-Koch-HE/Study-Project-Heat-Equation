\chapter{Herleitung der Wärmeleitungsgleichung}

Die Herleitung der Wärmeleitungsgleichung basiert auf dem Energieerhaltungssatz von Mayer, Joule und Helmholtz.
Dieser besagt, dass sich die im Inneren eines Körpers gespeicherte Energie
nur dadurch ändern kann, dass Energie über dessen Rand hinein- oder hinausfließt.
Zur Anwendung dieses Prinzips wird ein eindimensionales Stabelement betrachtet,
das ein Intervall von $[x-\Delta x,\,x+\Delta x]$ hat.
Die Temperaturverteilung im Stab wird durch die Funktion $u(x,t)$ beschrieben.

Die innere Energie pro Volumen ist proportional zur Temperatur und ergibt sich zu
$\rho c u(x,t)$, wobei $\rho$ die Dichte des Materials und $c$ die spezifische
Wärmekapazität bezeichnet.
Die gesamte innere Energie des betrachteten Stabelements erhält man daher durch
Integration über das Volumen.

Die zeitliche Änderung der inneren Energie über ein Zeitintervall $2\Delta t$ \cite[S.~13--16]{cannon1984one}
kann somit durch
\[
c\rho \int_{x-\Delta x}^{x+\Delta x}
\bigl(u(\xi,t+\Delta t)-u(\xi,t-\Delta t)\bigr)\,\mathrm{d}\xi
\]
beschrieben werden.
Dieser Ausdruck stellt die Energiebilanz des Stabelements dar und entspricht der
Anwendung des Energieerhaltungssatzes auf das betrachtete Kontrollvolumen.

Der Wärmetransport im Stab erfolgt durch Wärmeleitung und wird durch das
Fourier'sche Gesetz beschrieben.
In eindimensionaler Form lautet dieses
\[
q(x,t) = -k\,\frac{\partial u}{\partial x}(x,t),
\]
wobei $q(x,t)$ den Wärmestrom und $k$ die Wärmeleitfähigkeit des Materials bezeichnet.
Das negative Vorzeichen drückt aus, dass Wärme stets in Richtung abnehmender
Temperatur fließt.

Der Netto-Wärmestrom in das Stabelement ergibt sich aus der Differenz der
Wärmestrome an den Randpunkten $x-\Delta x$ und $x+\Delta x$.
Setzt man diesen Wärmestrom in die Energiebilanz ein und lässt die Größen
$\Delta x$ und $\Delta t$ gegen null gehen, so erhält man eine lokale
Differentialgleichung für die Temperaturentwicklung:
\[
\rho c\,\frac{\partial u}{\partial t}
=
\frac{\partial}{\partial x}
\left(
k\,\frac{\partial u}{\partial x}
\right).
\]

Für konstante Materialparameter vereinfacht sich diese Gleichung zur
klassischen Wärmeleitungsgleichung
\[
\frac{\partial u}{\partial t}
=
\alpha\,\frac{\partial^2 u}{\partial x^2},
\qquad
\alpha = \frac{k}{\rho c},
\]
wobei $\alpha$ als thermische Diffusivität bezeichnet wird.

Als äquivalent geltende Notationen sind:

$$
\frac{\partial u}{\partial t}=a^2\frac{\partial^2 u}{\partial x^2} 
$$

$$
u_t(x,t) = a^2 u_{xx}(x,t)
$$

$$
u_t = a^2 u_{xx}
$$






Die Wärmeleitungsgleichung lässt sich auch direkt aus der Energiebilanz in einem infinitesimalen Volumenelement herleiten \cite[S.~15--28]{langemann2018so}.
Dabei wird der Energieerhaltungssatz mit dem Fourierschen Gesetz kombiniert, um die partielle Differentialgleichung für die Temperaturentwicklung abzuleiten.

\subsubsection{Energieerhaltung}
Betrachten wir ein kleines Stück eines Stabes im Intervall \([x, x + \Delta x]\).
Die in diesem Stück gespeicherte thermische Energie \(E\) ist proportional zur Temperatur \(u(x,t)\):
\[
E = \rho \, c_p \, u(x,t) \, \Delta x
\]
wobei \(\rho\) die Dichte, \(c_p\) die spezifische Wärmekapazität und \(u(x,t)\) die Temperatur am Ort \(x\) zur Zeit \(t\) bezeichnet.

\subsubsection{Wärmestrom nach Fourier}
Der Wärmestrom \(I\) wird durch das Fouriersche Gesetz beschrieben:
\[
I = -k \, u_x
\]
wobei \(k\) die Wärmeleitfähigkeit ist.

\subsubsection{Netto-Wärmefluss und Energiebilanz}
Der Netto-Wärmefluss in das Intervall \([x, x + \Delta x]\) ergibt sich aus der Differenz der Wärmeströme an den Grenzen:
\[
I(x) - I(x + \Delta x) \approx -I_x \Delta x
\]
Nach dem Energieerhaltungssatz ist die zeitliche Änderung der Energie im Volumenelement gleich dem Netto-Wärmefluss:
\[
\rho c_p u_t \Delta x = - I_x \Delta x
\]
Durch Kürzen von \(\Delta x\) erhält man:
\[
\rho c_p u_t = - I_x
\]

\subsubsection{Einsetzen des Fourier'schen Gesetzes}
Setzt man das Fouriersche Gesetz für \(I\) ein, ergibt sich:
\[
I = -k u_x
\]

\[
I_x = - (k u_x)_x
\]

Damit wird die Energiebilanz zu:
\[
\rho c_p u_t = (k u_x)_x
\]

\subsubsection{Vereinfachung für homogene Materialien}
Für homogene Materialien (konstante Wärmeleitfähigkeit \(k\)) vereinfacht sich die Gleichung zu:
\[
u_t = \frac{k}{\rho c_p} u_{xx}
\]
Die Konstante
\[
a^2 = \frac{k}{\rho c_p}
\]
wird als thermische Diffusivität bezeichnet. Die Wärmeleitungsgleichung lautet somit:
\[
u_t = a^2 u_{xx}
\]


%So einfach ist Mathematik, Partielle Differentialgleichungen für Anwender. Springer, Abschnitte 2.1--2.3.

