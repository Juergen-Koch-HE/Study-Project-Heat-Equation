\chapter{Mathematisches Modell und Problemformulierung}


Konkret befasst sich diese Arbeit mit einer homogen eindimensionalen Wärmeleitgleichung, welche keine Wärmequelle hat und deren Dirichlet-Randbedingungen eine einheitliche konstante Temperatur von 0°C haben. In diesem Fall beschriebt die Gleichung, ein eindimesionales Stabelement, in einer perfekten Laborumgebung, beim Erkalten. Sie soll dem Betrachter in der Simulation später visuell vermitteln wie sich das Material im labor verhalten würde. Um diese Problem zu lösen wurde die allgemeine Wärmeleitungsgleichung ohne Wärmequelle benutzt: \[u_t = a^2\,u_{xx}\]

Dabei steht $u_t$ für den nach der Zeit abgeleiteten Teil der Gleichung und $u_{xx}$ für den nach dem Ort abgeleiteten Teil der Gleichung $a^2$ beschreibt in dieser Gleichung den spezifischen Temperaturleitungskoeffizient, der beschreibt, wie gut oder schlecht das Material Wärme leitet. 

\subsection{Wichtig}
Anfangstemperatur U(x,0) = …
