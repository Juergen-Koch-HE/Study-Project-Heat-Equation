\chapter{Theoretische Grundlagen}
In diesem Abschnitt werden Theoretische Grundlagen geschaffen, auf denen die späteren Kapitel aufbauen.

\section{Wärmeleitung als physikalisches Phänomen}
Das Phänomen der Wärmeleitung beschreibt den Wärmefluss in einem Körper als Wärmetransport von warmen zu kälteren Bereichen \cite[S.~12--27]{langemann2018so}. Es beschreibt, wie sich die Wärmeenergie ohne ohne makroskopische Bewegung des Materials (Konvektion) ausbreitet.Diese Projekt befasst sich ausschließlich mit dem eindimensionalen Fall (1D), also zum Beispiel einem Stab, bei dem die Temperatur nur entlang einer Raumkoordinate variiert.Ausgehend von einem isotropen homogenen Material wird Wärme im Inneren des Körpers durch Temperaturunterschiede von warmen zu kälteren Bereichen transportiert, sodass sich die Temperatur mit der Zeit ausgleicht. Die Temperatur, die dadurch an einem Ort $x$ zu einer Zeit $t$ in diesem Material entsteht, wird durch $u(x,t)$ beschrieben. Daraus ergibt sich die Temperaturfunktion $u(x,t)$ mit $t \geqq 0$ und $x \in (a,b)$, wobei $a$ und $b$ die Endpunkte des eindimensionalen Körpers sind. Sie beschreibt für jeden Punkt zu jeder Zeit die jeweilige Temperatur. Beide Randpunkte dieses Körpers sind damit $x=a$ und $x=b$. Die Randpunkte des Materials sind gesondert zu betrachten. Sie können entsprechend der Situation angepasst werden, je nach Modellannahme kann über die Randpunkte Wärme mit der Umgebung ausgetauscht werden oder nicht. Das bedeutet, dass der Körper seine Wärme an die Umgebung abgeben oder aus ihr aufnehmen kann oder (idealisiert) thermisch isoliert ist. Realistisch ist, dass der Körper mit seiner Umgebung Wärme austauscht. Jedoch kann in der Mathematik auch mit perfekt isolierten Randpunkten gearbeitet werden. Im Großen und Ganzen gibt es drei Typen von Randbedingungen für die Wärmeleitungsgleichung. Beginnend mit der Dirichlet-Randbedingung nach Johann Peter Gustav Lejeune Dirichlet. Im eindimensionalen Fall besteht der Rand aus genau zwei Punkten, nämlich $x=a$ und $x=b$. Auch hier kann zu jeder Zeit an beiden Randpunkten die Temperatur festgelegt werden, als $u(a,t)=q(a,t)$ und $u(b,t)=q(b,t)$ oder der Einfachheit halber kann konstanter Wert angenommen werden, zum Beispiel 20°C. Dies könnte der Raumtemperatur entsprechen, in der sich der Körper befindet. Nachfolgend die Neumann-Randbedingung nach Carl Gottfried Neumann, sie gibt anstatt der Außentemperatur den Wärmestrom über die Randpunkte vor. Der Wärmestrom über die Randpunkte wird dabei so formuliert $-k \cdot u_x(a,t) = I(a,t)$ und $-k \cdot u_x(b,t) = I(b,t)$ mit $t \geqq 0$. Dabei ist $k \ge 0$ die Wärmeleitfähigkeit des Körpers und $I(a,t)$,$I(b,t)$ sind die vorgegebenen Wärmeströme an den Randpunkten. Zum Schluss die Robin-Randbedingung nach Victor Gustave Robin. Sie ist die realistischste Randbedingung, denn sie kombiniert Eigenschaften der Dirichlet- und Neuman-Randbedingung. In der Robin-Randbedingung wird der Wärmestrom an den Randpunkten berücksichtigt und eine Temperaturdifferenz zwischen der Umgebungstemperatur am Rand q(x,t) und der Temperatur des Körpers am Rand $u(x,t)$ angesetzt, also an den Randpunkten bei $x=a$ und $x=b$. Außerdem beinhaltet sie noch eine weitere Komponente, den Wärmeübergangskoeffizienten $\beta(x)$, welcher beschreibt, wie stark Wärme zwischen dem Körper und der Umgebung über den Rand ausgetauscht wird. Der Wärmeübergangskoeffizienten bestimmen wie schnell Wärme von dem Körper oder System an seine Umgebung abgegeben wird. Die Wärmeleitfähigkeit $k$ beschreibt dagegen den Wärmetransport im Inneren des Materials, während $\beta(x)$ den Austausch über die Ränder beschreibt. Daraus ergibt sich folgende Gleichung: $I(t,x) = β(x)* [U(t,x) - Q(t,x)]$ für $t \geqq 0$ $x \in \{a,b\}$.
%(So einfach ist Mathematik, xxxx, S.12-27).

\section{Modellannahme und physikalische Größen}
Konkret befasst sich diese Arbeit mit einem eindimensionalen Stabelement, dessen Wärmestrom ausschließlich entlang der X-Achse verläuft. Das Material des Stabelements wird als homogen vorausgesetzt, sodass die Stoffwerte ortsunabhängig sind. Es gibt weder externe noch interne Wärmequellen. Es herrschen Dirichlet-Randbedingungen mit einer einheitlich konstanten Temperatur von 0°C. In diesem Fall beschreibt die Gleichung ein eindimensionales Stabelement in einer perfekten Laborumgebung beim Erkalten. Sie soll dem Betrachter in der Simulation später visuell vermitteln, wie sich das Material im Labor verhalten würde. Um diese Probleme zu lösen, wurde die allgemeine Wärmeleitungsgleichung ohne Wärmequelle \cite[S.~15]{cannon1984one} herangezogen: \[u_t(x,t) = a^2\,u_{xx}(x,t)\]

Dabei steht $u_t$ für den nach der Zeit abgeleiteten Teil der Gleichung und $u_{xx}$ für den nach dem Ort abgeleiteten Teil der Gleichung. $a^2$ beschreibt in dieser Gleichung den spezifischen Temperaturleitkoeffizient, der beschreibt, wie gut oder schlecht das Material Wärme leitet. 


