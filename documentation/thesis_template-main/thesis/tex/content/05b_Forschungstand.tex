\chapter{Momentaner Forschungsstand zur Wärmeleitung}
Ein Experiment des Helmholtz-Zentrums Dresden-Rossendorf, der Universität Bonn und des Centre national de la recherche scientifique hat sich mit der Wärmeleitfähigkeit von Halbmetallen befasst \cite{doi:10.1073/pnas.2408546122}. Es wurde ein Halbmetall, Zirkonium-Pentatellurid (ZrTe5), einer Temperatur in der Nähe des absoluten Nullpunkts -273,15 °C ausgesetzt sowie einem sehr hohen Magnetfeld. Das führte im Gegensatz zur Annahme der Forscher dazu, dass sich die Schwingungen der Phononen und die Bewegung der Elektronen koppelten und so der Wärmeleitwert für das Halbmetall anfing, periodisch stark im Magnetfeld zu oszillieren. %(B. Bermond, 2025, xxx)

