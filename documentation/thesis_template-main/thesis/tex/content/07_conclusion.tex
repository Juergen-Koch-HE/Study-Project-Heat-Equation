\chapter{Fazit und Ausblick}
Fundamental gelerntes umschließt, wie eine physikalische Gesetzmäßigkeit, die Energieerhaltung, in eine partielle Differentialgleichung übersetzt wurde und diese wiederum durch numerische Diskretisierung in einem Python-Skript verarbeitet wurde.

Die Arbeit mit den Fourier-Reihen, zeigt wie die Wahl von der Anzahl der Sinuswellen (N) einen sehr großen Einfluss hat auf die beiden Phänomene Gibbs und Überschwingungen. Eine Reduktion dieser Phänomene ist wiederum wünschenswert für die Genauigkeit. Die Annäherung ist jedoch sehr gut von Fourier-Reihen und dadurch wird verständlich, wieso es Anwendung findet in der Praxis.

Durch die interaktive Aufbereitung der Aufgaben wurde deutlich, das komplexe physikalische Prozesse begreifbar gemacht werden können. Eine schrittweise Reduktion der Komplexität, wie das weglassen von Wärmequellen und die eindimensionale Betrachtung, wie es bei dem Labor der Fall ist hilft dabei.

\bigskip

Der nächste logische Schritt wäre nun die mathematische Lösung und Simulation der 2D-Wärmeleitung, womit man den realen Bedingungen schon ein klein wenig näher kommt. Dies war auch ein Ziel dieses Studienprojekts, konnte allerdings leider nicht erfüllt werden.

Die Einbindung einer externen oder internen Wärmequelle wäre eine Möglichkeit aktive Heizvorgänge darzustellen. Statt fester Temperaturen an den Enden sind auch variable Temperaturen ein weiterer möglicher Schritt.
\bigskip


Abschließend ist zu sagen, dass die auseinandersetzung der physikalischen und mathematischen Betrachtung wie sich Wärme in einem Festkörper verhält aufschlussreich war und neue Betrachtungsweisen geliefert hat für alltägliche Prozesse, auch wenn in diesem Zuge auch die eigenen mathematischen Grenzen klar wurden. 