\chapter{Fazit und Ausblick}
Im Rahmen dieses Studienprojekts wurde nachvollzogen, wie eine physikalische Gesetzmäßigkeit, die Energieerhaltung, in eine partielle Differenzialgleichung übersetzt wurde und wie sich diese anschließend analytisch lösen sowie in einem Python-Skript auswerten und visualisieren lässt.

Die Arbeit mit den Fourier-Reihen zeigt, wie die Wahl der Anzahl der Sinuswellen (N) einen sehr großen Einfluss auf die beiden Phänomene Gibbs und Überschwingungen hat. Eine Reduktion dieser Phänomene ist wiederum wünschenswert für die Genauigkeit. Die Annäherung ist sehr gut von Fourier-Reihen und dadurch wird verständlich, wieso es Anwendung findet in der Praxis.

Durch die interaktive Aufbereitung der Aufgaben wurde deutlich, dass komplexe physikalische Prozesse begreifbar gemacht werden können. Eine schrittweise Reduktion der Komplexität, wie das Weglassen von Wärmequellen und die eindimensionale Betrachtung, wie es bei dem Labor der Fall ist, hilft dabei.

Der nächste logische Schritt wäre nun die mathematische Lösung und Simulation der 2D-Wärmeleitung, womit man den realen Bedingungen schon ein klein wenig näher kommt. Dies war auch ein Ziel dieses Studienprojekts, konnte allerdings leider nicht erfüllt werden.

Die Einbindung einer externen oder internen Wärmequelle wäre eine Möglichkeit, aktive Heizvorgänge darzustellen. Statt fester Temperaturen an den Enden sind auch variable Temperaturen ein weiterer möglicher Schritt.

Zusätzlich war vorgesehen, eine 2D Simulation der Wärmeleitungsgleichung auf einem vorhandenen Hardware-Cluster der Hochschule bereitzustellen, um eine performante Ausführung zu ermöglichen. Dieses Ziel wurde nicht erreicht.

Abschließend lässt sich festhalten, dass die Auseinandersetzung mit der physikalischen und mathematischen Beschreibung der Wärmeleitung in Festkörpern aufschlussreich war und neue Perspektiven auf alltägliche Vorgänge eröffnet hat.