\chapter{Analyse und Methodik}\label{sec:design}

Frage: Wie gehst du vor, um dieses Modell zu untersuchen / zu lösen?

Jetzt kommt das Wie, nicht das Was.

Bei dir z. B.:

Trennung der Variablen

Produktansatz

Eigenwertproblem

analytische Lösung

numerische Diskretisierung

Vergleich analytisch ↔ numerisch

🧠 Wichtig:

Hier erklärst du Methoden

nicht „wir programmieren jetzt“, sondern wie man theoretisch zur Lösung kommt

➡️ Das ist die Denk- und Rechenarbeit.





Konkret befasst sich diese Arbeit mit einer homogen eindimensionalen Wärmeleitglei-chung, welche keine Wärmequelle hat und deren Ränder eine einheitlich konstante Temperatur haben. Das bedeutet diese Gleichung einen Körper, in beschriebenem Fall, eine infinitesimal dünne Metallschiene beim Erkalten in einem perfekten Labor Umfeld beschreibt. 
Allgemeine Wärmeleitgleichung ohne Wärmequelle: Ut` = Ux`` * a^2
Dabei steht Ut’ für den nach der Zeit abgeleiteten Teil der Gleichung und Ux`` für den nach dem Ort abgeleiteten Teil der Gleichung a beschreibt in dieser Gleichung den spe-zifischen Temperaturleitungskoeffizient, der beschreibt, wie gut oder schlecht das Mate-rial Wärme leitet.
Auf dieser Basis aufbauend wurden Dirichlet-Randbedingungen von 0°Grad Celsius angenommen. 






\section{Theorie}

\subsection{Labor Übersicht}

Das Labor für das Modul Modellbildung und Simulation soll den Studenten schrittweise die Lösung der Wärmeleitgleichung nach dem Produktverfahren nähergebracht werden.

1.	Imports (sollen gleich bleiben, wie bei den anderen Laboren).

2.	Modelling: Als Einleitung zum Thema, wird ein Video eingefügt dieses Video beschreibt das Verhalten des Materials, wenn es an einer Seite durch eine Wärmequelle erhitzt wird. 

3.	Darauf folgt die Herleitung der Wärmeleitgleichung aus dem Energieerhaltungs- Fouriers- und Gaußgesetz. 

4.	Danach muss der Student seine erste Aufgabe lösen diese beinhaltet das Berechnen des Temperaturleitkoeffizienten. Die notwendigen Informationen sind durch Wikipedia Links bereitgestellt.

5.	Lösen der Wärmeleitgleichung für Stab in 1D ohne Wärmequelle. Explizite Lö-sung mit Anfangs- und Randbedingungen.

6.	Lösen der Wärmeleitgleichung für Stab in 1D mit Wärmequelle. Explizite Lö-sung mit Anfangs- und Randbedingungen.

7.	Simulation/Diskretisierung

Zusätzliche Aufgaben, für den Fall, dass wir schneller fertig werden.
Berechnen der Lösung für die Wärmeleitgleichung auf näherungsweisem Numerischen weg.

\subsection{Werkzeuge und Infrastruktur}
Die technische Umsetzung erfolgt in der Programmiersprache Python unter Verwen-dung von:
-	Jupyter Notebook für interaktive Entwicklung
-	NumPy für numerische Berechnungen
-	Matplotlib für Visualisierung der Ergebnisse
-	Pandas für die strukturierung der berechneten ergebnise
-	Math für das nutzen von variablen wie pi, cos, sin, ...
Die Anwendung soll abschließend auf einem Hochschul-Cluster bereitgestellt werden. Die Kompatibilität mit der dortigen Infrastruktur (Python-Version, Bibliotheken, evtl. Containerisierung) wird im Projektverlauf berücksichtigt.

\subsection{Mathische Grundlagen}

Bevor die Wärmeleitgleichung aufgestellt wird, werden in diesem Abschnitt wichtige mathematische Grundlagen behandelt, die einzelne  vorgänge auf dem Lösungsweg erst möglich machen. 

\subsubsection{Differeentialgleichung}
\subsubsubsection{Exponentialansatz}
\subsubsubsection{Linearkombination von Lösungen}
Die Theroie der Linearkombination von Lösungen homogener Differentialgleichungen besagt dass, sobald diese mindestens zwei lösungen hat, 
diese mit belibigen Konstanten multipliziert und dann addiert werden dürfen und dass sich daraus dann wiedere eine Lösung ergibt. 
Durch die Linearkombination der Lösungen von y_1 und y_2 ergibt sich fogkendes Phänomen.
\[ Y(x) = C_1*y_1(x)+C_2*y_2(x)\]

Da eine homogene DGL immer Y(x) = 0 sein muss, ergibt sich aus der Liniarität der Teillösungen auch also y_1=0 und Y_2=0 sein.
Da eine homogene Differentialgleichung per Definition erfüllt ist, wenn ihr Gesamtwert Null ergibt, müssen auch die Teillösungen \(y_{1}\) und \(y_{2}\) beim Einsetzen jeweils das Ergebnis Null liefern. Aufgrund der Linearität übertragen sich die Konstanten auf diese Einzelergebnisse, sodass die gesamte Linearkombination ebenfalls Null ergibt:
\[ 0 = C_1*0+C_2*0\] 
Somit ist bewiesen, dass die kombinierte Funktion \(Y(x)\) die Bedingung der homogenen Gleichung weiterhin erfüllt.

\subsubsubsection{Characteristische Gleichung}


\subsubsection{Seperation der Variablen}
\subsubsection{Gausche Zahlenebene}
\subsubsection{Fourier}





\subsection{Analytische Lösungsverfahren}
Es wurde das analytische Lösungsverfahren zur Lösung der Wärmeleitgleichung durchgeführt. 
!!!! Vorteile von Lösung mit Analytischen verfahren!!!! 
Randbedingungen 
\[U(0,t) =  0;		t \ge 0,\; 0 < x < \pi
U(Pi,t) = 0; \]

Anfangswert 
\[U(x,0) = 1,  \]

Es wurde der der Ansatz der Separation der variablen verwendet, um den Ort \[X(x)\] und die Zeit \[T(t)\] zu trennen. Dadurch wird die Gleichung in zwei Bereiche aufgeteilt und die Gleichungen können nach der jeweiligen Unbekannten aufgelöst werden, 
Wärmeleitgleichung: \[Ut `= a^2 * Ux``\]

Ableiten nach der Zeit 
\[U(x,t) = X(x) * T(t) -> Ut`(x,t) = X(x) * T`(t)\]

Ableiten nach dem Ort X(x)
\[U(x,t) = X(x) * T(t) -> Ux``(x,t) = X``(x) * T(t)\] 

Einsetzen in die Wärmeleitgleichung 
\[X(x) * T`(t) = a^2 * X``(x) * T(t)\]

Durchführung der Separation der Variablen
\[( X``(x) / X(x) ) = ( T`(t) / ( a^2 * T(t) ) = q   q = Konst.\] (!!!mathematischen lambda trick erklären!!!!))

Auflösen der Wärmeleitgleichung nach dem Ort X(x)
\[( X``(x) / X(x) ) = q\]  

Um die lineare homgene Differentialgleichung in die Charakteristische Gleichung umzuformen wird der Exponential ansatz \[X(x) = e^(lambda*x)\] genutzt.  Koch buch S.505
\[0 = -X``(x) + w *  X(x)
0 = -\lambda^2*e^(\lambda x) + w*e^(\lambda x) 
0 = -\lambda^2 + w             
\lambda1,2 = +/- \sqrt{w}\]

Einsetzen der Lösung in die Standardgleichung 
\[X(x) = C1 * e^(x*\sqrt{w}) + C2 * e^(-x * \sqrt{w})\]

Ableiten nach der Zeit T(t)
\[( T`(t) / ( a^2 * T(t) ) ) = w \]

Umformen in die Charakteristische Gleichung
\[T`(t) = a^2 * T(t) * w
0 = -T`(t) + a^2 * T(t) * w
0 = -\lambda + a^2 * w 
\lambda = a^2 * w\]

Einsetzen der Lösung in die Standardgleichung
\[T(t) = C3 * e^(t*a^2*w)\]

Zusammen führen der separierten Gleichungen, Allgemeine Gleichunng
\[U(x,t)=(C1*e^(x*\sqrt{w}) + C2 *e^(-x*\sqrt{w}))*C3*e^(a^2 * w)\]

Einsetzen der Randbedingungen  in \[U(0,t) =  0\], Ortsteil der Gleichungen 
\[X(0) = C1*e^ (0*\sqrt{w})+ C2*e^ (-0*\sqrt{w}) = 0 \]

Einesetzen der Randbedingungen in in den Ortsteilder Gelchungen. 
\[U(0,t) = u(\pi,t) = 0 \quad \Longrightarrow \quad X(0) = X(\pi) = \]

Aufgrund der Linearkombination kann C bestimmt werden. 
\[X(0) = C1*e^0*\sqrt{w} + C2*e^-(0*\sqrt{w}) = C1*1 + C2*1 = 
C1 + C2 = 0 => C1 = -C2 => |C1| = |C2| = c\]

\[X(x) = C*e^(x*\sqrt{w}) - C*e^(-x*\sqrt{w}) = C*( e^(x*\sqrt{w}) - e^(-x*\sqrt{w}) ) \]

\[X(0) = C_1 e^{0\sqrt{\omega}} + C_2 e^{-0\sqrt{\omega}}\]
\[X(0) = C_1 \cdot 1 + C_2 \cdot 1\]
\[C_1 + C_2 = 0\]
\[C_1 = -C_2\]
\[C_1| = |C_2| = C\]
\[X(x) = C e^{x\sqrt{\omega}} - C e^{-x\sqrt{\omega}}\]


\[X(\pi) = 0 = C * ( e^(pi*\sqrt{w}) - e^(-pi*\sqrt{w})) )  C \neq 0\] 
\[=> e^(pi*\sqrt{w}) - e^(-pi*\sqrt{w}) = 0\] 
\[=> e^(pi*\sqrt{w}) = e^(-pi*\sqrt{w})\]
\[=> e^(2*pi*\sqrt{w}) = 1\]

Wegen Gausscher Zahlenebene \[e^(2*pi*\sqrt{w}) = e^(i*2*pi*\sqrt{w})\]

\[=> 2'\pi*\sqrt{w} = i*2*pi*k       k ∈ Z \{0}\]
\[=> \sqrt{w} = i*k                  k ∈ Z \{0}\]
\[=> w = -k^2                        k ∈ Z \{0}\]


Einsetzen in \[ \lambda = a \pm i b  X(x) = e^(a*x) + (C1*cos(bx) + C2*sin(bx))\]

\[X(x) = C1*cos(kx) + C2*sin(kx)
 X(x) = C_1 \cos(kx) + C_2 \sin(kx) \]
\[X(0) = C1*cos(0) + C2*sin(0) = C1*1 + C2*0 = C1 = 0
 X(0) = C_1 \cos(0) + C_2 \sin(0) = C_1 \cdot 1 + C_2 \cdot 0 = C_1 = 0 \]

\[X(x) = C2*sin(kx)
 X(x) = C_2 \sin(kx) \]
\[X(\pi) = C2*sin(k*pi) = 0  C2 != 0 => sin(k*pi) = 0 => k ∈ Z \{0} \]    ???? richtig 
\[
X(\pi) = C_2 \sin(k\pi) = 0 C_2 \neq 0 \;\Longrightarrow\; \sin(k\pi) = 0 k \in \mathbb{Z} \setminus \{0\}
\]

Zusammenführung der Lösung mit Randbedingungen
U(x,t) = C2*sin(kx)*C3*e^(-a^2*k^2*t)
U(x,t) = Ck*sin(kx)*e^(-a^2*k^2*t)         Ck = C2*C3 = konst.
\[ U(x,t) = C_2 \sin(kx)\, C_3 e^{-a^2 k^2 t}
U(x,t) = C_k \sin(kx)\, e^{-a^2 k^2 t}  C_k = C_2 C_3 = \text{konst.} \]

Die Anfangsbedingung \[U(x,0) = 1\] wird durch die Fourier-Analyse gelöst, um die Koeffizienten \[C_k\] zu bestimmen.
Ck = 2/pi * Integral von 0 bis pi von U(x,0)*sin(kx) dx
Ck = 2/pi * Integral von 0 bis pi von 1*sin(kx) dx
\[
C_k = \frac{2}{\pi} \int_{0}^{\pi} U(x,0)\sin(kx)\,\mathrm{d}x
C_k = \frac{2}{\pi} \int_{0}^{\pi} 1 \cdot \sin(kx)\,\mathrm{d}x
\]
\[ C_k = \frac{2}{\pi} \int_{0}^{\pi} \left( -\frac{1}{k}\cos(kx) \right) \,\mathrm{d}x \]

\[ C_k = \frac{2\,(1-\cos(k\pi))}{\pi k} \]     \[ \cos(k\pi) = (-1)^k, \]
\[ C_k = \frac{2\,(1-(-1)^k)}{\pi k} \]

Einsetzen von geraden und ungeraden zahlen 2 und 3 
\[
C_k    = \frac{2\,(1-(-1)^2)}{\pi\cdot 2}
       = \frac{2\,(1-1)}{2\pi}
       = 0,
\]
\[
  C_k = \frac{2\,(1-(-1)^3)}{\pi\cdot 3}
      = \frac{2\,(1+1)}{3\pi}
      = \frac{4}{3\pi}.
\]

Daraus folgt für ungerade \(k\) gilt 
=> \[C_k = 4/(pi*k)\]

und für gerade \(k\) 
=> \[C_k = 0\]

Die spezielle Lösung der Wärmeleitungsgleichung lautet somit:
\[
U(x,t) = \frac{4}{\pi k}\,\sin(kx)\,e^{-a^2 k^2 t}.
\]

\section{Gibbisch Phänomen}
\section{Überschwinger Glätten}






\section{Praxis}





