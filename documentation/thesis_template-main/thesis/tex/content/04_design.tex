\chapter{Analyse und Methodik}\label{sec:design}

\section{Mathematische Grundlagen}
Bevor die Wärmeleitungsgleichung aufgestellt wird, werden in diesem Abschnitt einige wichtige mathematische Grundlagen behandelt, die einzelne Vorgänge auf dem Lösungsweg in der analytischen Lösung genauer beschreiben. 

\subsection{Differenzialgleichungen}
Eine Differenzialgleichung ist eine Gleichung, deren Lösung eine Fuktion ist und die neben der unbekanten Funktion auch mindestens eine Ableitung dieser Funktion enthält.
Es gibt außer gewöhnlichen auch noch andere Gleichungsarten. Die für dieses Projekt wichtige, ist die partielle Differenzialgleichung, bei der die gesuchte Funktion von mehreren Variablen abhängig ist und partielle Ableitungen enthält.  
Die Wärmeleitungsgleichung ist eine partielle Differenzialgleichung, 
da sie partielle Ableitungen nach der Zeit und nach dem Ort enthält \cite[S.~481--483]{koch2015mathematik}. %(Koch \& Stempfle, 2018, S.481-483).

\subsubsection{Linearkombination von Lösungen}
Das Prinzip der Linearkombination von Lösungen, linearer homogener Differenzialgleichungen, besagt, dass sobald diese mindestens zwei Lösungenn hat, diese mit beliebigen Konstanten multipliziert und dann addiert werden dürfen und dass sich daraus dann wieder eine Lösung ergibt. 
Durch die Linearkombination der Lösungen von $y_1$ und $y_2$ ergibt sich folgendes Phänomen:
\[ y(x) = C_1 \cdot y_1(x)+C_2 \cdot y_2(x)\]

Dieses Prinzip ermöglicht es, aus Lösungen der linearen homogenen Differenzialgleichung die allgemeine Lösung zu formulieren \cite[S.~501]{koch2015mathematik}. %(Koch \& Stempfle, 2018, S.501).

\subsubsection{Charakteristische Gleichung}
Das Prinzip der Charakteristischen Gleichung wird beim Lösen linearer homogener Differenzialgleichungen mit konstanten Koeffizienten angewendet und erleichtert das Auflösen der Gleichung. Dabei wird der Exponentialansatz dazu genutzt: $y(x) = e^{\lambda x}$ 
Dabei gilt beim Lösen der Gleichung für die Ableitung wird  $y^k(x) = \lambda^k \cdot e^{\lambda x}$. Dabei kann man den gemeinsamen Faktor kürzen, so dass $y^k(x) = \lambda^k$ entsteht.
Dabei gilt, dass $y^0(x) = \lambda^0 = 1 $ und $y^1(x)=\lambda \cdot e^{\lambda x}$ ist.
Hat die Gleichung nun den Grad n, entsteht ein charakteristisches Polynom. Dieses muss aufgelöst werden, um die Eigenwerte für $\lambda_1$ bis $\lambda_n$ zu berechnen.

Es gibt vier verschiedene Arten von Eigenwerten, zwischen denen unterschieden werden muss. 
Es gibt einfache und mehrfache reelle Eigenwerte sowie einfache und mehrfache komplexe Eigenwerte.

Aus den Lösungen der einfachen reellen Eigenwerte werden Eigenwertfunktionen der Form 
$C \cdot e^{\lambda x}$. 
Die Werte von $\lambda_1$ bis $\lambda_n$ werden in die allgemeine Lösung:
\[
C_1 \cdot e^{\lambda_1 x}+ ... + C_n \cdot e^{\lambda_n x}
\]
eingesetzt \cite[S.~506--512]{koch2015mathematik}.

Bei mehrfachen reellen Eigenwerten treten zusätzlich Faktoren der Form 
$x^m e^{\lambda x}$ auf.

Die Lösungen der komplexen Eigenwerte treten paarweise in der Form 
$\lambda = a \pm i b$ auf. Dabei steht $a$ für den reellen Teil, 
$i$ für die imaginäre Einheit und $b$ für den Faktor des imaginären Teils 
\cite[S.~454]{langemann2018so}.

Für einfache komplexe Eigenwerte ergibt sich die Lösung \cite[S.~509--512]{koch2015mathematik}:
\[
e^{a x} \left(C_1 \cos(bx) + C_2 \sin(bx)\right)
\]


Bei mehrfachen komplexen Eigenwerten treten zusätzlich Faktoren der Form
\[
x^m e^{a x}\left(C_1 \cos(bx) + C_2 \sin(bx)\right)
\]
auf.





\subsection{Fourier-Reihen}
Eine Fourier-Reihe stellt eine periodische mathematische Funktion, in sich wiederholenden Sinus- und Kosinus Termen dar. Das bedeutet die Ergebnisse dieser Gleichungen wiederholen sich nach einer Periode $T$.
Damit eine Funktion als periodisch gilt, sollte sie $f(x+p)=f(x)$ sein, wobei die Periode $p > 0$ ist \cite[S.~191]{koch2015mathematik}. 
%(Koch \& Stempfle, 2018, S.191).
Eine Fourier-Reihe wird so definiert: 

$$f(t)=\frac{a_0}{2}+\sum_{k=1}^{\infty}\left(a_k \cdot \cos(k\cdot w \cdot t)+b_k \cdot \sin(k\cdot w\cdot t)\right),\qquad w =\frac{2\pi}{T}$$ 

Dabei steht $\frac{a_0}{2}$ für den Mittelwert der Funktion über eine Periode. $a_k$ und $b_k$ sind die Fourier-Koeffizienten. $k$ indexiert die Frequenzteile der Reihe, sie kompensiert die überschwingenden positiven und negativen Flächenanteile der Funktion und führt zu einem Glättungseffekt des Ergebnisses, je höher $k$ ist. $w$ die Kreisfrequenz, die bestimmt, wie schnell die Funktion innerhalb einer Periode schwingt, und $t$ die Zeitvariable \cite[S.~578--581]{koch2015mathematik}. 
%(Koch \& Stempfle, 2018, S.578-581).

Die Fourier-Reihe funktioniert für Funktionen, die auf dem betrachteten Intervall stückweise stetig sind. Dort wo die Funktion stetig ist, konvergiert die Fourier-Reihe dann gegen den Funktionswert. An Sprungstellen hingegen konvergiert sie gegen den Mittelwert der links- und rechtsseitigen Grenzwerte \cite[S.~577--582]{koch2015mathematik}.

Ob eine Funktion $f(t)$ ungerade oder gerade ist, erkennt man, wenn man sie auf Achsen symetrie prüft. Ist die Funktion $f(-t) = f(t)$, dann ist sie achsensymetrisch. Ist die Funktion $f(-t) = -f(t)$, dann ist sie punktsymetrisch und somit ungerade \cite[S.~188--189]{koch2015mathematik}. %(Koch \& Stempfle, 2018, S.188-189).

\[a_k = \frac{2}{T}\int_{-\frac{T}{2}}^{\frac{T}{2}} f(t) \cdot \cos(k\cdot w \cdot t)\,dt, \qquad k = 0,1,2,\ldots\]
\[b_k = \frac{2}{T}\int_{-\frac{T}{2}}^{\frac{T}{2}} f(t) \cdot \sin(k\cdot w \cdot t)\,dt, \qquad k = 1,2,\ldots\]


\subsection{Analytisches Lösungsverfahren}

Es wurden die Dirichlet-Randbedingungen $U(0,t)=0$ und $U(\pi,t)=0$ mit $t \ge 0$ angenommen. Diese Wahl verienfacht die die Lösung des Ortsteils im weitern Verlauf, da aus $U(0,t)=0$ die Bedinugn $X(0)= 0$ folgt und dadurch in der allgemeinen Ortslösung der Cosinusanteil verschwindet. $U(\pi,t)=0$ führt zu $X(\pi)=0$ und damit auf einfache Eigenwerte, da $\sin(k\cdot \pi)=0$ für $k \in \mathbb{N}$ gilt.

Als Anfangswert wurde ein konstanter Wert von $1$ im Inneren ausgewählt, so entsteht in Kombination mit den Randbedingungen von $U(0,t)=U(\pi,t)=0$ ein Gefälle aus dem inneren des Körpers zum Rand hin.
Der Anfangswert wurde als Konstante gewählt, da so die Berechnung des Fourier-Koeffizienten $b_k$ unkompliziert ist. Außerdem ist das Gibbssche Phänomen nahe der Sprungstellen zu beobachten.
Der Anfangswert ist stückweise definiert, damit an den Rändern keine Widersprüche entstehen.
\[U(x,0) = \qquad \begin{cases} 1, 0 < x < \pi, \\ 0, x=0, \text{ oder } x=\pi. \end{cases}\] 


Es wurde der der Produktansatz gewählt um die Wärmeleitungsgleichung aufzustellen. Dadurch wird angenommen das man die Lösung der Gleichung in der Form $u(x,t)=X(x) \cdot T(t)$ schreiben kann. Beginnend mit der Separation der Variablen, um den Ort $X(x)$ und die Zeit $T(t)$ zu trennen. Dadurch wird die Gleichung in zwei Bereiche aufgeteilt und diese können nach der jeweiligen Unbekannten aufgelöst werden.

Wärmeleitungsgleichung hat die Form $u_t(x,t)= a^2 \cdot u_{xx}$ um diese Form aus dem Produktansatz zu erzeugen,
wird der Produktansatz einmal nach der Zeit ableiten:  
\[u(x,t) = X(x) \cdot T(t) \Rightarrow  u_t(x,t) = X(x) \cdot T_t(t)\]

Und zweimal nach dem Ort ableiten:
\[u(x,t) = X(x) \cdot T(t) \Rightarrow u_{xx}(x,t) = X_{xx}(x) \cdot T(t)\] 

Die Gleichungen die durch das Ableiten nach dem Ort und der Zeit entstehen, werden nun in die Wärmeleitungsgleichung eingesetzt:
\[X(x) \cdot T_t(t) = a^2 \cdot X_{xx}(x) \cdot T(t)\]

Um die Seperation der Variablen durchzuführen, muss diese Gleichung so umgeformt werden, das alle Terme, die von der Zeit $t$ abhängen, auf einer Seite stehen und alle Terme, die vom Ort $x$ abhängen, auf der anderen Seite. Das wird erreicht, in dem diese Gleichung durch den Term $X(x) \cdot T(t) \cdot a^2$ geteilt wird. Es kürzen sich der jeweilige Orts und Zeit Term aus einer Seite und es entsteht die folgende Gleichung:
\[\frac{ X_{xx}(x)}{ X(x)}  =  \frac{T_t(t)}{a^2 \cdot T(t) }\]

Durch die Separation der Variablen entstehen zwei unabhängige Gleichgungsteile, nach dem Ort $\frac{ X_{xx}(x)}{ X(x)}$ und der Zeit $\frac{T_t(t)}{a^2 \cdot T(t) }$. Dadurch könnte man den Ort verändern und würde keinen Einfluss auf die Zeit nehmen und umgekehrt. Das bedeutet mathematisch, dass der linke Term nur vom Ort x und der rechte Term nur von der Zeit t abhängt. Damit die Gleichheit für alle $x$ und $t$ gilt, müssen beide Terme gleich derselben Konstante $w$ sei. Das bringt den Vorteil, dass die Wärmeleitungsgleichung, welche eine partielle Differenzialgleichung ist, in zwei gewöhnliche Differenzialgleichungen geteilt wird. Die einmal nach dem Ort und einmal nach der Zeit getrennt sind \cite[S.~105--107]{langemann2018so}. %(Mathematik ..., xxxx, S.105-107)
\[\frac{ X_{xx}(x)}{ X(x)}  =  \frac{T_t(t)}{a^2 \cdot T(t) } = w \qquad   w = Konst.\] 

Auflösen des Ortsteils X(x) der Wärmeleitungsgleichung.
\[\frac{X_{xx}(x)}{X(x)} = w\]  

Um die lineare homogene Differenzialgleichung in die charakteristische Gleichung umzuformen, wird der Exponentialansatz $X(x) = e^{\lambda \cdot x}$ genutzt (Koch \& Stempfle, 2018, S.505).
\[0 = -X_{xx}(x) + w \cdot  X(x)\]
\[0 = -\lambda^2\cdot e^{\lambda x} + w\cdot e^{\lambda x} \]
\[0 = -\lambda^2 + w \]            
\[\lambda_{1,2} = \pm \sqrt{w}\]

Einsetzen von $\lambda_{1,2}$ und dadurch erzeugen der Eigenwertfunktionen für den Ortsteil der Wärmeleitungsgleichung. 
\[X(x) = C_1 \cdot e^{x\cdot\sqrt{w}} + C_2 \cdot e^{-x \cdot \sqrt{w}}\]

Auflösen des Zeitteils T(t) der Wärmeleitungsgleichung.
\[\frac{T_t(t)}{ a^2 \cdot T(t)} = w \]

Umformen in die Charakteristische Gleichung.
\[T_t(t) = a^2 \cdot T(t) \cdot w\]
\[0 = -T_t(t) + a^2 \cdot T(t)\cdot w\]
\[0 = -\lambda + a^2 \cdot w\] 
\[\lambda = a^2 \cdot w\]

Einsetzen von $\lambda$ und dadurch erzeugen der Eigenwertfunktionen für den Zeitteil der Wärmeleitungsgleichung. 
\[T(t) = C_3 \cdot e^{t \cdot a^2 \cdot w}\]

Zusammenführen der separierten Gleichungen. Erzeugen der allgemeinen Lösung.
\[u(x,t)=(C_1 \cdot e^{x \cdot\sqrt{w}} + C_2\cdot e^{-x\cdot \sqrt{w}})\cdot C_3 \cdot e^{t \cdot a^2 \cdot w}\]

Einsetzen der Randbedingungen in den Ortsteil der Gleichungen. 
\[u(0,t) = u(\pi,t) = 0 \quad \Longrightarrow \quad X(0) = X(\pi) = 0 \]
\[X(0) = C_1 \cdot e^{0\sqrt{w}} + C_2 \cdot e^{-0\cdot \sqrt{w}}\]
\[X(0) = C_1 \cdot 1 + C_2 \cdot 1\]
\[C_1 + C_2 = 0\]
\[C_1 = -C_2\]
\[|C_1| = |C_2| = C\]
\[X(x) = C_1 \cdot e^{x \cdot \sqrt{w}} - C_2 \cdot e^{-x \cdot \sqrt{w}} \Rightarrow C \cdot ( e^{x \cdot \sqrt{w}} - e^{-x \cdot \sqrt{w}}) \]
Aufgrund der Linearkombination von Lösungen werden $C_1$ und $C_2$ zu $C$ zusammengefasst.

\[X(\pi) = C \cdot ( e^{\pi \cdot \sqrt{w}} - e^{-\pi \cdot \sqrt{w}} )\, \qquad  C \neq 0\] 
\[\Rightarrow e^{\pi \cdot \sqrt{w}} - e^{-\pi \cdot \sqrt{w}} = 0\] 
\[e^{\pi \cdot \sqrt{w}} = e^{-\pi \cdot \sqrt{w}}\]
\[e^{2 \cdot \pi \cdot \sqrt{w}} = 1\]

Für die Exponentialfunktion gilt $e^{z} = 1$ daraus folgt $z=2\cdot \pi \cdot i \cdot k$ mit $k \in \mathbb{Z} \setminus \{0\}$.

\[\Rightarrow 2\cdot \pi \cdot \sqrt{w} = i \cdot 2 \cdot \pi \cdot k
\]
\[\Rightarrow \sqrt{w} = i \cdot k
\]
\[\Rightarrow w = -k^{2}
\]

Damit gilt  $\lambda = \pm \sqrt{w} = \pm ik$ also a = 0 und b = k. diese Werte werden in die Eigenwertfunktion eingesetzt.  
\[X(x) = e^{a\cdot x}  \cdot (C_1 \cdot \cos(bx) + C_2 \cdot \sin(bx))\]
\[X(x) = C_1 \cdot \cos(k\cdot x) + C_2 \cdot \sin(k\cdot x) \]

Einsetzen der Randbedingung $X(0)=0$
\[X(0) = C_1 \cdot \cos(0) + C_2 \cdot \sin(0) = C_1 \cdot 1 + C_2 \cdot 0 = C_1 = 0 \]
\[X(x) = C_2 \cdot \sin(k\cdot x)\]
Einsetzen der Randbedingung $X(\pi)=0$
\[X(\pi) = C_2 \cdot \sin(k \cdot \pi) = 0, \qquad C_2 \neq 0, \qquad k \in \mathbb{Z} \setminus \{0\}\]
\[\sin(k \cdot \pi) = 0\]    


Bildung der Produktlösung.
\[u(x,t) = C_2\cdot sin(k \cdot x) \cdot C_3 \cdot e^{-a^2 \cdot k^2 \cdot t}, \qquad C_k := C_2C_3= \text{Konst.} \]
\[u(x,t) = C_k \cdot sin(k \cdot x) \cdot e^{-a^2 \cdot k^2 \cdot t}\]         

\subsection{Von der Produktlösung zur Fourier-Reihe}
Aufgrund der homogenen Dirichlet-Randbedingungen 
$u(0,t)=u(\pi,t)=0$ ergeben sich als Eigenfunktionen des
Ortsteils ausschließlich Sinusfunktionen. Cosinusfunktionen können die Randbedingungen nicht erfüllen. Aus gegebenem Grund wird nur der Sinus-Koeffizient $b_k$ beachtet, währen der Cosinus-Koeffizient $a_k$ entfällt.

Die Anfangsbedingung $u(x,0) = 1$ wird durch die Fourier-Analyse behandelt, um den Koeffizienten $b_k$ zu bestimmen.
\[
b_k = \frac{2}{\pi} \int_{0}^{\pi} u(x,0)\sin(k \cdot x)\,\mathrm{d}x\]
\[
b_k = \frac{2}{\pi} \int_{0}^{\pi} 1 \cdot \sin(k \cdot x)\,\mathrm{d}x
\]

\[ b_k = \frac{2\cdot \,(1-\cos(k \cdot \pi))}{\pi \cdot k} \]     \[ \cos(k \cdot \pi) = (-1)^k, \]
\[ b_k = \frac{2\cdot \,(1-(-1)^k)}{\pi \cdot k} \]

Einsetzen von geraden und ungeraden Zahlen $k = 2$ und $k = 3$. 
\[
b_k    = \frac{2\cdot \,(1-(-1)^2)}{\pi\cdot 2}
       = \frac{2\cdot \,(1-1)}{2 \cdot \pi}
       = 0,
\]
\[
  b_k = \frac{2\cdot \,(1-(-1)^3)}{\pi\cdot 3}
      = \frac{2\cdot \,(1+1)}{3 \cdot \pi}
      = \frac{4}{3 \cdot \pi}.
\]

Daraus folgt, für ungerade $k$ gilt:
 \[b_k = \frac{4}{\pi \cdot k}\]

Für gerade $k$:
 \[b_k = 0\]

Die spezielle Lösung der Wärmeleitungsgleichung mit Fourier-Reihe für $k=2n+1$ lautet somit:
\[
u(x,t) = \sum_{k=1}^{\infty} \frac{4}{\pi \cdot k}\cdot \,\sin(k \cdot x)\cdot \,e^{-a^2 \cdot k^2 \cdot t}
\]




\section{Gibbssches Phänomen}
Das Gibbssche Phänomen, entdeckt von Josiah Willard Gibbs, bezeichnet die Über- und Unterschwingungen, die bei der Approximation einer Funktion durch ihre Fourier-Reihe in der Nähe einer Sprungstelle auftreten (Koch \& Stempfle, 2018, S.581-582). Das kann man sich als einen abrupten Richtungswechsel in einem Graphen vorstellen (Koch \& Stempfle, 2018, S.258). Zum Beispiel bei einer Rechteckfunktion, die linksseitig für $x < 1$ mit $f(x) = 1$ und rechtseitig für als $x > 1$ mit$ f(x) = -1$ definiert ist, entsteht dort wo sich die Intervalle berühren bei $x = 1$ eine Sprungstelle.
Das Gibbssche Phänomen ist eine feste Größe und hat eine Abweichung von der Sptze des Überschwingers zur Spitze des Unterschwingers von ungefähr $18\%$. Durch das Einbeziehen sehr vieler Sinus- und Kosinusglieder wird der Bereich in dem das Gibbssche Phänomen auftritt schmaler, jedoch sinkt die Abweichung Grenzbereich nicht unter $18\%$ \cite[S.~581--582]{koch2015mathematik}. %(Koch \& Stempfle, 2018, S.581-582). 

Dieses Phänomen tritt besonders in Unstetigen Grenzregionen von Differenzialgleichungen auf und hat somit eine relevanz für die Wärmeleitglichung




