\chapter{Analyse und Methodik}\label{sec:design}

Frage: Wie gehst du vor, um dieses Modell zu untersuchen / zu lösen?

Jetzt kommt das Wie, nicht das Was.

Bei dir z. B.:

Trennung der Variablen

Produktansatz

Eigenwertproblem

analytische Lösung

numerische Diskretisierung

Vergleich analytisch ↔ numerisch

Wichtig:

Hier erklärst du Methoden

nicht „wir programmieren jetzt“, sondern wie man theoretisch zur Lösung kommt

Das ist die Denk- und Rechenarbeit.





Konkret befasst sich diese Arbeit mit einer homogen eindimensionalen Wärmeleitglei-chung, welche keine Wärmequelle hat und deren Ränder eine einheitlich konstante Temperatur haben. Das bedeutet diese Gleichung einen Körper, in beschriebenem Fall, eine infinitesimal dünne Metallschiene beim Erkalten in einem perfekten Labor Umfeld beschreibt. 
Allgemeine Wärmeleitgleichung ohne Wärmequelle: \[u_t = a^2\,u_{xx}\]
Dabei steht Ut’ für den nach der Zeit abgeleiteten Teil der Gleichung und Ux`` für den nach dem Ort abgeleiteten Teil der Gleichung a beschreibt in dieser Gleichung den spe-zifischen Temperaturleitungskoeffizient, der beschreibt, wie gut oder schlecht das Mate-rial Wärme leitet.
Auf dieser Basis aufbauend wurden Dirichlet-Randbedingungen von 0°Grad Celsius angenommen. 



\section{Mathematische Grundlagen}

Bevor die Wärmeleitgleichung aufgestellt wird, werden in diesem Abschnitt wichtige mathematische Grundlagen behandelt, die einzelne  vorgänge auf dem Lösungsweg erst möglich machen. 

\subsection{Differentialgleichung}
Eine Differeenzialgleichung ist eine Gleichung deren Lösung eine Fuktion ist. Sie hat außer der unbekannten Funktion auch mindestens eine Ableitung dieser Funktion in sich.
So könnte eine gewöhliche Differeenzialgleichung aussehen $f(x) = f'(x)+1$. Es gibt auser der gewöhnlichen auch noch andere Gleichungs arten eine für diesen 
Bericht wichtige, ist die partielle Differeenzialgleichung. Sie beinhaltet im gegensatz zur gewöhlichen mehrer Veränderliche. 
Ein beispiel für eine partielle Differeenzialgleichung ist $2*f(x,y) = f_{yy}(x,y)$. Auch die Wärmeleitgleichung ist eien partielle Differeenzialgleichung 
da sie einersetz nach der Zeit und anderseits nach dem Ort abgeleitet ist (Koch \& Stempfle, 2018, S.481-483).

\subsubsection{Exponentialansatz}

\subsubsection{Linearkombination von Lösungen}
Das Prinzip der Linearkombination von Lösungen, linearer homogener Differentialgleichungen, besagt dass sobald diese mindestens zwei lösungen hat, 
diese mit beliebigen Konstanten multipliziert und dann addiert werden dürfen und dass sich daraus dann wieder eine Lösung ergibt. 
Durch die Linearkombination der Lösungen von $Y_1$ und $Y_2$ ergibt sich folgendes Phänomen:
\[ Y(x) = C_1*Y_1(x)+C_2*Y_2(x)\]

Da das Ergebnis der Anwendung einer homogenen Differentialgleichung immer gleich null sein muss, sonst wäre sie nicht homogen. Ergibt sich durch die Linearität auch dass, die Teillösungen $Y_1$ und $Y_2$ beim einsetzen in die Gleichung $Y(x)$ gleich null ergeben.
\[ 0 = C_1*0+C_2*0\] 

Dieses Prinzip ermöglicht es aus den einfachen Lösungen der linearen homogenen Differetntialgleichung eine Allgemeine lösung zu formulieren 
(Koch \& Stempfle, 2018, S.501)

\subsubsection{Charakteristische Gleichung}
Das Prinzip der Charakteristischen Gleichung wird beim lösen linearer homogener Differentialgleichungen mit konstanten Koeffizienten angwendet und erleichtert das Auflösen der Gleichung. Dabei wird der Exponentialansatz dazu genutzt: $Y(x) = e^{\lambda x}$

Dabei gilt beim lösen der Gleichung die Ableitung wird  $Y^k(x) = \lambda^k e^{\lambda x}$. Dabei kann man den gemeinsamen Faktor kürzen so das $Y^k(x) = \lambda^k$ entsteht.
Dabei gilt das $Y^0(x) = \lambda^0 = 1 $ und $Y^1(x)=\lambda$ ist.
Hat die Gleichung nun den Grad n, entsteht ein Characterristisches Polynom. Welches aufgelöst werden muss, um die Eigenwerte für $\lambda_1$ bis $\lambda_n$ zu berechnen.
Folgendes Beispiel beschreibt den Vorgang.
\[ 
Y_{xx}(x)+2*Y_{x}-Y(x)= 0 \Rightarrow\ \lambda^2+2*\lambda-1 = 0
\]

\[
\lambda_{1,2}=\frac{-2\pm\sqrt{2^2-4\cdot1\cdot(-1)}}{2\cdot1}
=\frac{-2\pm\sqrt{8}}{2}\]
\[\lambda_{1}=\frac{-2+\sqrt{8}}{2}=  -1 + \sqrt{2}\]
\[\lambda_{2}=\frac{-2+\sqrt{8}}{2}=  -1 - \sqrt{2}\]



Es gibt vier verschieden Arten von Eigenwerten, zwischen denen unterschieden werden muss. Es gibt einfache und mehrfache relle Eigenwerte sowie einfache komplexe und mehrfache komplexe Eigenwerte.
Aus den Lösungen der einfachen oder mehrfache reelle Eigenwerte werden Eigenwertfunktionen. Die  Werte von $\lambda_1$ bis $\lambda_n$ werden in die Eigenwertfunktion $C_1*e^{\lambda_1*x}+ ... + C_n*e^{\lambda_n*x}$ eingesetzt (Koch \& Stempfle, 2018, S.506-512).
Die Lösungen der komplexen Eigenwerte kommen mit einem komplexen so wie einem rellen anteil $\lambda = a \pm i*b$ dabei steht a für den rellen und i für die immaginäre Einheit welche b als den immaginären Anteil markiert (Koch \& Stempfle, 2018, S.454).
Die Lösungen der einfachen und mehrfachen koplexen Eigenwerte werden für $\lambda_1$ bis $\lambda_n$ in diese Eigenwertfunktionen $e^{a*x}*C_1*cos(b*x) + e^{a*x}*C_2*sin(b*x)+ ... + e^{a*x}*C_n*cos(b*x)$ (Koch \& Stempfle, 2018, S.509-512).

\[Y(x)= C_1*e^{0,9*x}+C_2*e^{-1,9x}\]  fehler in der logik hier beheben !!!!!!!!!!!!!!!!
\[
Y(x)=\left(C_1+C_2x\right)e^{-x}
\]


\subsection{Seperation der Variablen}

\subsection{Gausche Zahlenebene} grmmatik korregieren !!!!!!!!!!!!!!!

Die Gausche Zahlenebene beschreibt die komplexen Zahlen. Eine komplexe Zahl besteht aus einem real und imaginären mit der Einheit $i$ Teil.
$z = x +iy$ Beschreibt die Reelle Zahl wobei man sich $x$, den Realteil und $y$, den Imaginärteil als zwei Koordinaten eines Punktes in einem Kartesischen Koordinatensytems vorstellen muss. Wobei die horizontale Achse den Realteil und die vertikal Achse den Imaginärteil beschreibt. Die komplexen Zahlen können in verschiedenen formen geschrieben werden, hier wird im folgenden die Exponentilaform besprochen. Diese sieht folgend aus $ z = r e^{i \phi}$, $r$ beschreibt die länge des Vektors und $\phi$ den Phasenwinkel, dadurch bewegt sich eine komplexe zahl immer im Kreis um den Nullpunkt des Koordinatensystems. Das einzige das sich durch das rechnen it diesen Zahlen verändern kann ist der abstand zum zentrum des Koordinatensystems, die komplexe Zahl wird immer nur ziwschen den Quadranten wander. 

Die eulersche identität (entdeckt von Leonhard Euler) die für jede reele komplexe Zahl gilt. Beschreibt wie sich die Zahlen im Einheitskreis um den Nullpunkt drehen.
$e^{i \phi} = \cos(\phi) + i sin(\phi)$
$|e^{i \phi}| = 1$ das bedeuted das der Betrag der komplexen Zahlen, die die Form $e^{i \phi}$ haben, immer den Abstand 1 zum Nullpunkt des Koordinatensystems. Da sie sich auf dem Einheitskreis bewegen und dieser in jede richtung und in jedem winkel den selben Betrag hat ist. (Koch \& Stempfle, 2018, S.451-456)

\subsection{Fourier-Reihen}

Eine Fourier-Reihe stellt eine periodische mathematische Funktion, in sich wiederholenden Sinus- und Kosinus thermen dar. Das beduted die ergebnisse dieser Gleichungen wiederholen sich nach einer Periode $T$.
Damit eine Funktion als periodisch gilt sollte sie $f(x+p)=f(x)$ sein, wobei die Periode $p > 0$ ist (Koch \& Stempfle, 2018, S.191).

Eine Fourier-Reihe wird so definiert $f(t)=\frac{a_0}{2}+\sum_{k=1}^{\infty}\left(a_k\cos(kwt)+b_k\sin(kwt)\right),\qquad \omega=\frac{2\pi}{T}$. Dabei steht $\frac{a_0}{2}$ für den Mittelwert der Funktion über eine Periode. $a_k$ und $b_k$ sind die Fourier-Koeffizienten, sie bestimmen die Form der Schwingung der Reihe, ob sie eher Wellenförmig oder Eckig ist. $k$ indexiert die Frequenzteile der Reihe, sie kompensiert die die überschwingenden positiven und negativen Flächenanteile der Funktion und führt zu einem glettungs effekt des Ergebnis, je höher $k$ ist. $w$ die Kreisfrequenz die bestimmt wie schnell die Funktion innerhalb einer Periode schwingt und $t$ die Zeitvariable (Koch \& Stempfle, 2018, S.578-581).
Um die Fourier Koeffizienten $a_k$ und $b_k$ zu berechnen muss man wissen ob eine periodische Funktion gerade, also einen Kosinusanteil, oder ungerade, also einen Sinusanteil hat, ist. Ob eine funktion $f(t)$ ungerade oder gerade ist erkennt man wenn man sie auf Achsen symetrie prüft. Ist die funktion $f(-t) = f(t)$ dann ist sie Achsensymetrisch. Is die Funktion $f(-t) = -f(t)$ dann ist sie Punktsymetrisch und somit ungerade(Koch \& Stempfle, 2018, S.188-189).

\[a_k = \frac{2}{T}\int_{-\frac{T}{2}}^{\frac{T}{2}} f(t)\cos(k\omega t)\,dt, \qquad k = 0,1,2,\ldots\]
\[b_k = \frac{2}{T}\int_{-\frac{T}{2}}^{\frac{T}{2}} f(t)\sin(k\omega t)\,dt, \qquad k = 1,2,\ldots\]


\subsection{Analytische Lösungsverfahren}
Es wurde das analytische Lösungsverfahren zur Lösung der Wärmeleitgleichung durchgeführt. 
!!!! Vorteile von Lösung mit Analytischen verfahren!!!! 

Randbedingungen: 
\[
U(0,t)=0 \;\text{und}\; U(\pi,t)=0, \qquad t \ge 0,\; 0 < x < \pi
\]


Anfangswert 
\[U(x,0) = 1,  \]

Es wurde der der Ansatz der Separation der variablen verwendet, um den Ort $X(x)$ und die Zeit $T(t)$ zu trennen. Dadurch wird die Gleichung in zwei Bereiche aufgeteilt und die Gleichungen können nach der jeweiligen Unbekannten aufgelöst werden, 
Wärmeleitgleichung: \[U_t= a^2 * U_{xx}\]

Ableiten nach der Zeit 
\[U(x,t) = X(x) * T(t) \Rightarrow  U_t(x,t) = X(x) * T_t(t)\]

Ableiten nach dem Ort X(x)
\[U(x,t) = X(x) * T(t) \Rightarrow U_{xx}(x,t) = X_{xx}(x) * T(t)\] 

Einsetzen in die Wärmeleitgleichung 
\[X(x) * T_t(t) = a^2 * X_{xx}(x) * T(t)\]

Durchführung der Separation der Variablen
\[( X_{xx}(x) / X(x) ) = ( T_t(t) / ( a^2 * T(t) ) = q   q = Konst.\] (!!!mathematischen lambda trick erklären!!!!))

Auflösen der Wärmeleitgleichung nach dem Ort X(x)
\[( X_{xx}(x) / X(x) ) = q\]  

Um die lineare homgene Differentialgleichung in die Charakteristische Gleichung umzuformen wird der Exponential ansatz \[X(x) = e^{\lambda*x}\] genutzt.  Koch buch S.505
\[0 = -X_{xx}(x) + w *  X(x)
0 = -\lambda^2*e^{\lambda x} + w*e^{\lambda x} 
0 = -\lambda^2 + w             
\lambda_{1,2} = +/- \sqrt{w}\]

Einsetzen der Lösung in die Standardgleichung 
\[X(x) = C_1 * e^{x*\sqrt{w}} + C_2 * e^{-x * \sqrt{w}}\]

Ableiten nach der Zeit T(t)
\[( T_t(t) / ( a^2 * T(t) ) ) = w \]

Umformen in die Charakteristische Gleichung
\[T_t(t) = a^2 * T(t) * w
0 = -T_t(t) + a^2 * T(t) * w
0 = -\lambda + a^2 * w 
\lambda = a^2 * w\]

Einsetzen der Lösung in die Standardgleichung
\[T(t) = C3 * e^{t*a^2*w}\]

Zusammen führen der separierten Gleichungen, Allgemeine Gleichunng
\[U(x,t)=(C_1*e^{x*\sqrt{w}} + C_2 *e^{-x*\sqrt{w}})*C_3*e^{a^2 * w}\]

Einsetzen der Randbedingungen \[U(0,t) =  0\] 

in, Ortsteil der Gleichungen 
\[X(0) = C1*e^{0*\sqrt{w}}+ C2*e^{-0*\sqrt{w}} = 0 \]

Einesetzen der Randbedingungen in in den Ortsteil der Gleichungen. 
\[U(0,t) = u(\pi,t) = 0 \quad \Longrightarrow \quad X(0) = X(\pi) = 0 \]

Aufgrund der Linearkombination kann C bestimmt werden. 
\[X(0) = C1*e^{0*\sqrt{w}} + C2*e^{-0*\sqrt{w}} = C_1*1 + C_2*1 = 
C_1 + C_2 = 0 => C_1 = -C_2 => |C_1| = |C_2| = c\]

\[X(x) = C*e^{x*\sqrt{w}} - C*e^{-x*\sqrt{w}} = C*( e^{x*\sqrt{w}} - e^{-x*\sqrt{w}}) \]

\[X(0) = C_1 e^{0\sqrt{\omega}} + C_2 e^{-0\sqrt{\omega}}\]
\[X(0) = C_1 \cdot 1 + C_2 \cdot 1\]
\[C_1 + C_2 = 0\]
\[C_1 = -C_2\]
\[C_1| = |C_2| = C\]
\[X(x) = C e^{x\sqrt{\omega}} - C e^{-x\sqrt{\omega}}\]


\[X(\pi) = 0 = C * ( e^{\pi*\sqrt{w}} - e^{-\pi*\sqrt{w}} )  C \neq 0\] 
\[\Rightarrow e^{\pi*\sqrt{w}} - e^{-pi*\sqrt{w}} = 0\] 
\[\Rightarrow e^{\pi*\sqrt{w}} = e^{-pi*\sqrt{w}}\]
\[\Rightarrow e^{2*\pi*\sqrt{w}} = 1\]

Wegen Gausscher Zahlenebene \[e^{2*pi*\sqrt{w}} = e^{i*2*pi*\sqrt{w}}\]    korregieren !!!!!

\[\Rightarrow 2\pi*\sqrt{w} = i*2*pi*k,\qquad k \in \mathbb{Z} \setminus \{0\}
\]
\[\Rightarrow \sqrt{w} = i*k, \qquad k \in \mathbb{Z} \setminus \{0\}
\]
\[\Rightarrow w = -k^{2}, \qquad k \in \mathbb{Z} \setminus \{0\}
\]


Einsetzen in \[ \lambda = a \pm i b  X(x) = e^{a*x} + (C_1*cos(bx) + C_2*sin(bx))\]

\[X(x) = C_1*cos(kx) + C_2*sin(kx)
 X(x) = C_1 \cos(kx) + C_2 \sin(kx) \]
\[X(0) = C_1*cos(0) + C_2*sin(0) = C_1*1 + C_2*0 = C1 = 0 \]
\[X(0) = C_1 \cos(0) + C_2 \sin(0) = C_1 \cdot 1 + C_2 \cdot 0 = C_1 = 0 \]

\[X(x) = C_2*sin(kx)
 X(x) = C_2 \sin(kx) \]
\[X(\pi) = C_2*sin(k*pi) = 0  C_2 != 0 => sin(k*pi) = 0 => \qquad k \in \mathbb{Z} \setminus \{0\} \]    ???? richtig 
\[
X(\pi) = C_2 \sin(k\pi) = 0 C_2 \neq 0 \;\Longrightarrow\; , \qquad k \in \mathbb{Z} \setminus \{0\}
\]

Zusammenführung der Lösung mit Randbedingungen:
\[U(x,t) = C_2*sin(kx)*C_3*e^{-a^2*k^2*t}\]
\[U(x,t) = C_k*sin(kx)*e^{-a^2*k^2*t}\]         
\[Ck = C_2*C_3 = \text{konst.}\]
\[ U(x,t) = C_2 \sin(kx)\, C_3 e^{-a^2 k^2 t}
U(x,t) = C_k \sin(kx)\, e^{-a^2 k^2 t}  C_k = C_2 C_3 = \text{konst.} \]

Die Anfangsbedingung $U(x,0) = 1$ wird durch die Fourier-Analyse gelöst, um die Koeffizienten $C_k$ zu bestimmen.
\[C_k = 2/pi * \text{Integral von 0 bis pi von} U(x,0)*sin(kx) dx\]
\[C_k = 2/pi * \text{Integral von 0 bis pi von} 1*sin(kx) dx\]
\[
C_k = \frac{2}{\pi} \int_{0}^{\pi} U(x,0)\sin(kx)\,\mathrm{d}x\]
\[
C_k = \frac{2}{\pi} \int_{0}^{\pi} 1 \cdot \sin(kx)\,\mathrm{d}x
\]
\[ C_k = \frac{2}{\pi} \int_{0}^{\pi} \left( -\frac{1}{k}\cos(kx) \right) \,\mathrm{d}x \]

\[ C_k = \frac{2\,(1-\cos(k\pi))}{\pi k} \]     \[ \cos(k\pi) = (-1)^k, \]
\[ C_k = \frac{2\,(1-(-1)^k)}{\pi k} \]

Einsetzen von geraden und ungeraden zahlen 2 und 3 
\[
C_k    = \frac{2\,(1-(-1)^2)}{\pi\cdot 2}
       = \frac{2\,(1-1)}{2\pi}
       = 0,
\]
\[
  C_k = \frac{2\,(1-(-1)^3)}{\pi\cdot 3}
      = \frac{2\,(1+1)}{3\pi}
      = \frac{4}{3\pi}.
\]

Daraus folgt für ungerade $k$ gilt:
 \[C_k = 4/(pi*k)\]

und für gerade $k$:
 \[C_k = 0\]

Die spezielle Lösung der Wärmeleitungsgleichung lautet somit:
\[
U(x,t) = \frac{4}{\pi k}\,\sin(kx)\,e^{-a^2 k^2 t}
\]

\section{Gibbssches Phänomen}
Das Gibbisch Phänomen, entdeckt von Josiah Willard Gibbbs, bezeichnet die Über- und Unterschwinger, die bei der Approximation einer Funktion durch ihre Fourier-Reihe in der Nähe einer Sprungstelle auftreten(Koch \& Stempfle, 2018, S.581-582). Das kann man sich als einen abrupten Richtungswechsel in einem Graphen vorstellen(Koch \& Stempfle, 2018, S.258). Zum Beipiel bei einer Rechteckfunktion, die linksseitig für $x < 1$ mit $f(x) = 1$ und rechtseitigen für als $x > 1$ mit$ f(x) = -1$ definiert ist, entsteht dort wo sich die Intervalle berühren bei $x = 1$ eine Sprungstelle.
Das Gibbssche Phänomen ist eine feste größe und hat eine Abweichung von der Sptze des Überschwingers zur Spitze des Unterschwingers von ungefähr $18\%$. Durch das einbeziehen sehr vieler Sinus- und Kosinusglieder wird der Bereich in dem das Gibbssche Phänomen auftritt schmaler, jedoch wird die Abweichung Grenzbereich nicht unter $18\%$(Koch \& Stempfle, 2018, S.581-582). 

Dieses Phänomen tritt besonders in Unstetigen Grenzregionen von Differeentilgleichungen auf und hat somit eine relevanz für die Wärmeleitglichung

\section{Überschwinger Glätten}





