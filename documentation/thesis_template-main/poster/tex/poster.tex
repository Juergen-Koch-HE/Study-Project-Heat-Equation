\documentclass[20pt]{beamer}
%% Fill in the page size here. If the proportions of the fonts
%% are not satisfactory, change the scale parameter
\usepackage[orientation=portrait,size=a1,scale=1.4]{beamerposter}
\usepackage[english,ngerman]{babel}
\usepackage{multicol}
\usepackage{caption}
\usepackage{wrapfig}
\usepackage[backend=bibtex,style=ieee, citestyle=numeric-comp, sorting=nty, mincitenames=1, maxcitenames=1]{biblatex}
\usepackage{hyperref}

\newlength{\textpagepad}\setlength{\textpagepad}{1cm}
\setlength{\columnsep}{2\textpagepad} % for captions

\newenvironment{blockwithcontent}[1]{%
  \begin{block}{#1}%
    \hspace{\textpagepad}%
    \begin{minipage}{\textwidth-2\textpagepad}%
}{%
    \end{minipage}%
  \end{block}%
}

% Usage: \parallelminipages[<optional:vertical alignment>]{<left page width>}{<left content>}{<right content>}
\newcommand{\parallelminipages}[4][c]{%
  \begin{minipage}{\textwidth}%
    \begin{minipage}[#1]{#2-0.5\textpagepad}%
      #3%
    \end{minipage}%
    \hfill%
    \begin{minipage}[#1]{\textwidth-#2-0.5\textpagepad}%
      #4%
    \end{minipage}%
  \end{minipage}%
}

% Usage: \debugframebox{<any content whose bounds you want to visualize>}
\newcommand{\debugframebox}[1]{{\setlength{\fboxsep}{0pt}\setlength{\fboxrule}{0.01pt}\framebox{#1}}}


\bibliography{literature.bib}
\mode<presentation>{\usetheme{poster}}

\title[CryptoWatch]{CryptoWatch}
\subtitle{Analyzing the Diversity of Cryptographic Algorithms of \\[.4em]Internet-Connected Services}
\author{Stud Ent - \textit{sten42@hs-esslingen.de} - University of Esslingen}
\institute{Supervisors: Prof. Dr. Tobias Heer, Nils Lohmiller \{tobias.heer, nils.lohmiller\}@hs-esslingen.de}

\begin{document}
    % change the following line to "ngerman" for German style date and logos
    \selectlanguage{english}

    \begin{frame}
        \begin{blockwithcontent}{1. Motivation: State of Cryptography}
            \justifying
            \begin{wrapfigure}{r}{0.5\textwidth}
                \centering
                \includegraphics[width=0.5\textwidth]{images/infrastructure.png}
                \caption{Internet Infrastructure Visualization~\cite{Infrastruc}}
                \label{fig:Infrastructure}
            \end{wrapfigure}
            Internet-connected services use different cryptographic \textbf{algorithms} and \textbf{key lengths} to secure their communication.
            Services that implement TLS use \textbf{certificates}.  
            Over time, these certificates, algorithms, and key lengths change. 
            For example, since 2020, browsers require a shorter certificate validity period of at most one year.
            Cryptographic algorithms will be replaced if more efficient ones become available or if those become insecure.
            Services that use asymmetric algorithms are threatened by quantum computers.
            Therefore, \textbf{Post-Quantum Cryptography (PQC)} emerges to protect these systems in the future.

            However, not only will systems need to be protected in the future, current systems already need protection mechanisms. 
            One possible attack scenario is \textit{\textbf{save now decrypt later}}.
            In this scenario, an attacker stores asymmetrically encrypted data and assumes that he can decrypt it with the help of quantum computers in the future.
            Such attacks can only be prevented by considering which algorithms will still be secure in the future.
            A transition of all asymmetric algorithms to pure PQC or a combination of PQC and classical asymmetric cryptography takes time.
            The sooner the transition to PQC begins, the better~\cite{CCC}.
        \end{blockwithcontent}

        \begin{blockwithcontent}{2. Research Idea: How do Cryptographic Algorithms Change over Time?}
            \parallelminipages{.35\textwidth}{
                \begin{figure}[h!]
                    \centering
                    \includegraphics[width=\textwidth]{images/OpenSSH.pdf}
                    \caption{OpenSSH versions change over time, along with them different cryptographic algorithms are used}
                    \label{SSH}
                \end{figure}
            }{
                \justifying
                \begin{itemize}
                    \item Due to the reduced validity period of the certificates, they have to be renewed more frequently
                    \item We will investigate whether the \textbf{algorithms change during the renewal of certificates}
                    \item Are there any \textbf{regional differences/inconsistencies}?
                    \item Figure~\ref{SSH} shows how OpenSSH versions change over time. We will do the same but for certificates and cryptographic algorithms.
                    \item The transition towards PQC poses new challenges
                    \item Google and Cloudflare have shown the practical use of hybrid PQC and asymmetric cryptography~\cite{CloudflarePQC}.
                    \item We examine whether there are Internet services that \textit{already support PQC algorithms}
                \end{itemize}
            }
        \end{blockwithcontent}
    
        \begin{blockwithcontent}{3. Key Research Challenges}
            \parallelminipages{.65\textwidth}{
                \justifying
                \begin{itemize}
                    \item What should a \textbf{scanning pipeline} look like?
                    \item What is the \textbf{current state} of cryptographic algorithms?
                    
                    \begin{itemize}
                        \item Which cryptographic algorithms are commonly used?
                        \item Which key lengths are often used?
                        \item Have PQC algorithms already been implemented?
                    \end{itemize}

                    \item What changes after \textbf{certificate renewals}?
                    \item Are there regional differences?
                \end{itemize}
            }{
                \begin{figure}[h!]
                    \centering
                    \includegraphics[width=\linewidth]{images/Scanning.pdf}
                    \caption{Scanning Pipeline}
                \end{figure}
            }
        \end{blockwithcontent}

        \begin{blockwithcontent}{4. References}
            \AtNextBibliography{\small}
            \printbibliography
        \end{blockwithcontent}
    \end{frame}
\end{document}
